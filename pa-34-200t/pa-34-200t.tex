\documentclass{article}
\usepackage{geometry}
    \geometry{
        a5paper,
        portrait,
        % The Jeppesen plate appears to be closer to 0.25in. I think
        % that 0.5in is looking best for checklists. Compromising to
        % accomodate longer line lengths.
        margin=0.25in,
        % rmargin=0.375in,
        % headsep is the separation between header and text. footskip is
        % the separation between baseline of last line of text and
        % baseline of footer. The default is a bit larger. Setting these
        % to the ~line height pleases me, aesthetically.
        headsep=\baselineskip,
        footskip=\baselineskip,
        includehead,
        includefoot
    }
% NASA says:
% > The horizontal spacing between characters should be 25% of the
% > overall size and not less than one stroke width.
%
% The microtype documentation says:
% > Letterspaced fonts for which settings don’t exist will be spaced out
% > by the default of 0.1 em [...]
% AND
% > The amount is specified in thousandths of 1 em [...]
%
% So, we're scaling the default spacing by 25%, and then converting to
% housandths of an em (0.1 * 1000 * .25).
\usepackage[letterspace=25]{microtype}
% Used for the checklist frames.
\usepackage[many]{tcolorbox}
% For the preflight checklist square.
\usepackage{amssymb}
% For finer control over multi-column layouts.
\usepackage{multicol}
% For finer control over headers and footers.
\usepackage{fancyhdr}
% Used for degree symbol.
\usepackage{gensymb}
% Used for PDF ToC links.
\usepackage{hyperref}
% For printing the creation date of the document.
\usepackage{datetime2}
% Used for drawing patterns (e.g., the striped emergency procedure background).
\usepackage{tikz}
\usetikzlibrary{patterns,patterns.meta}
% Used for performance charts.
\usepackage{booktabs}
% Used for performing math inline.
\usepackage{xfp}
% Used for formatting numbers.
\usepackage{siunitx}
\sisetup{
    math-rm=\symup,
    detect-all,
    group-minimum-digits=4,
    group-separator={,}
}
% Improves positioning of tables and figures.
\usepackage{float}
% The following two packages all for sans-serif math.
\usepackage{sansmathfonts}
\usepackage[T1]{fontenc}
% https://tex.stackexchange.com/questions/84175/itemize-without-line-feed
\usepackage{enumitem}

% This is a macro that formats the checklist items and adds a new line.
\def\checkitem#1#2{
    #1\dotfill#2

}

% Set the default font family to sans-serif.
\renewcommand{\familydefault}{\sfdefault}

% NASA says:
% > The vertical spacing between lines should not be smaller than 25-33%
% > of the overall size of the font.
\renewcommand{\baselinestretch}{1.25}

% Configure the header and footer.
\pagestyle{fancy}
\fancyhf{}
\fancyhead[L]{PA-34-200T, Piper Seneca II}
\fancyhead[R]{\rightmark}
\fancyfoot[L]{v.\today}
\fancyfoot[R]{\thepage}

% We don't need numbered sections.
\setcounter{secnumdepth}{0}

\begin{document}

% Apply microtype tracking adjustments.
\lsstyle

% [...] you can say \raggedcolumns if you don’t want the bottom lines to
% be aligned. The default is \flushcolumns, so TEX will normally try to
% make both the top and bottom baselines of all columns align.
\raggedcolumns

% This is a macro to assist in the preflight checklist.
\def\todoitem#1{
    \item[$\square$] #1 \dotfill
}

\section{Preflight Checklist}
\renewcommand{\rightmark}{Preflight Checklist}


\subsection{Master Switch On}

\begin{itemize}
    \todoitem{Interior and exterior lights}
    \todoitem{Pitot heat}
\end{itemize}

\subsection{Walk Around}

\begin{itemize}
    \todoitem{Control surfaces and cables}
    \todoitem{Drain fuel tank sumps}
    \begin{itemize}
        \item[$\bullet$] Remove all water and sediment; verify proper fuel.
    \end{itemize}
    \todoitem{Check crossfeed drains}
    \todoitem{Propellers}
    \todoitem{Air inlets and alternator belt tension}
    \todoitem{Oil level (6-8 quarts)}
    \todoitem{No obvious oil or fuel leaks}
\end{itemize}

\subsection{Landing Gear}

\begin{itemize}
    \todoitem{Strut exposure ($3\frac{1}{2}''$ for main, $2 \frac{1}{2}''$ for nose)}
    \todoitem{Visual inspection of tires (55 PSI for main, 31 PSI for nose)}
    \todoitem{Visual inspection of brake blocks}
\end{itemize}

\subsection{Finishing Up}

\begin{itemize}
    \todoitem{Clean windshield}
    \todoitem{Prep cockpit}
    \todoitem{Engine times and Stratus}
\end{itemize}

% This is the box that surrounds the checklist items.
\newtcolorbox{checklist}[1]{
    colback=white,
    colframe=black,
    fonttitle=\centering\bfseries,
    adjusted title={#1},
    sharpish corners,
    phantom=\phantomsection,
    add to list={toc}{subsection}
}

\phantomsection
\addcontentsline{toc}{section}{Normal Procedures}
\renewcommand{\rightmark}{Normal Procedures}

\twocolumn

\begin{checklist}{Before Start}
    \checkitem{Walk around}{complete}
    \checkitem{Seat belts}{fasten}
    \checkitem{PIC}{establish}
    \checkitem{Passengers}{brief}
\end{checklist}

\begin{checklist}{Engine Start}
    \checkitem{Parking brake}{set}
    \checkitem{Fuel selector}{on}
    \checkitem{Alternate air}{off}
    \checkitem{Cowl flaps}{open}
    \checkitem{Alternators}{on}
    \checkitem{Master switch}{on}
    \checkitem{3 green, no red}{check}

    \begin{center}
        \emph{\hypertarget{both-engines}{Repeat for both engines}}
    \end{center}

    \checkitem{Mixture}{full rich}
    \checkitem{Prop}{full forward}
    \checkitem{Throttle}{full open}
    \checkitem{Prime}{2-3 sec.}
    \checkitem{Throttle}{$\frac{1}{2}''$}
    \checkitem{Clear/starter}{engage}
    \checkitem{Throttle}{1000 RPM}
    \checkitem{Oil pressure}{above 30 PSI}
    \checkitem{Load meter}{check}
\end{checklist}

\begin{checklist}{After Start}
    \checkitem{Avionics master}{on}
    \checkitem{Circuit breakers}{check}
    \checkitem{Garmin database}{check}
    \checkitem{Garmin self-test}{check}
    \checkitem{ATIS \& clearance}{recieve}
\end{checklist}

\begin{checklist}{Before Taxi}
    \checkitem{Transponder}{set}
    \checkitem{COM \& NAV}{set}
    \checkitem{Initial heading}{set}
    \checkitem{Exterior lights}{set}
\end{checklist}

\begin{checklist}{Taxi}
    \checkitem{Brakes}{check}
    \checkitem{Heading indicator}{±5°}
    \checkitem{Attitude indicator}{check}
    \checkitem{Turn coordinator}{check}
\end{checklist}

% Mike Busch and John Deakin have made some interesting points about the
% engine run up process. We're running so rich, and at such low power on
% the ground, it's hard to identity any real problems with the ignition
% system outside of a dead spark plug, a dead mag, or severe spark plug
% fouling.
% TBD: Finish this.
% See: https://www.avweb.com/flight-safety/pelicans-perch-19putting-it-all-together/
\begin{checklist}{Engine Run-Up}
    \checkitem{Mixtures}{full rich}
    \checkitem{Props}{full forward}
    % John Deakin says:
    % > The usual 1,700 RPM for running up most TCM engines (or 2,000
    % > RPM for most Lycomings) is NOT critical. I’ve seen pilots diddle
    % > and dawdle trying to get exactly 1,700 but all this does is heat
    % > the engine up for no good purpose. Plus or minus a couple
    % > hundred RPM won’t hurt a thing, so push it up to “about 1,700”
    % > or “about 2,000” and get on with it.
    % > [...]
    % > It is also becoming very clear that the mag check at low power
    % > (anything less than cruise power) is not very useful for
    % > catching problems; it’s nothing more than a quick check to catch
    % > major problems like severe plug fouling, a “hot mag,” or a dead
    % > plug, or cylinder. This was well-known in the big old radials,
    % > where mag checks are almost always performed at about 30″ MP,
    % > and up around 2,300 RPM (varies with model).
    \checkitem{Throttles}{1000 RPM}
    \checkitem{Manifold pressure lines}{drain}
    \checkitem{Prop}{check feathering}
    \checkitem{Throttles}{1900 RPM}
    \checkitem{Mags}{check L \& R}
    \centering{
        (max drop 150; max $\Delta$ 50)
        \\
    }
    \checkitem{Alternate air}{check}
    % When running up the Lance, Jim Cherry was taken aback when I
    % cycled the propellor at 1800 RPM. His reasoning was that it's
    % overkill, it doesn't need to be cycled that high. Unnessecary
    % strain on things, why do it? But, I forget the ballpark that he
    % reccomended. It may have been 1000-1500 RPM.
    \checkitem{Prop}{cycle}
    \centering{
        ($\downarrow$ rpm $\uparrow$ mp $\downarrow$ oil press.)
        \\
    }
    \checkitem{Vacuum}{4.5-5.2$''$Hg}
    \checkitem{Oil pressure \& oil temp.}{check}
    \checkitem{Fuel pressure}{check}
    \checkitem{CHTs}{check}
    \checkitem{Load meter}{check}
    \checkitem{Alternate static}{check}
    \checkitem{Annunciator panel}{check}
    \checkitem{Throttles}{idle}
\end{checklist}

\begin{checklist}{Before Takeoff}
    \checkitem{Flight controls}{check}
    \checkitem{Flight instruments}{check}
    \checkitem{Alternate air}{off}
    \checkitem{Cowl flaps}{open}
    \checkitem{Flaps}{set}
    \checkitem{Trim}{set}

    % In the future, consider creating takeoff/landing cards to do the hard work
    % before the flight. Then, all we need to do is read through pre-computed
    % things. Also, consider how to do parts of the briefing while still on the
    % ramp, rather than right in front of the runway.
    \begin{center}
        \emph{\hypertarget{departure-briefing}{Departure briefing}}
    \end{center}

    % TODO: Include some prompt to think about any possibility of
    % tailwind, extreme crosswind, or extreme temperature inversion.
    \checkitem{\hyperlink{runway-length-table}{Takeoff distance}}{brief}
    \checkitem{Terrain \& obstacles}{brief}
    \checkitem{Takeoff minimums}{brief}
    \checkitem{Departure procedure}{brief}

    \begin{center}
        \emph{Abnormal operations}
    \end{center}

    \checkitem{Rejected takeoff}{brief}
    \checkitem{Engine power loss}{brief}
\end{checklist}

\begin{checklist}{Takeoff}
    \checkitem{Time off}{note}
    \checkitem{Doors \& windows}{secure}
    \checkitem{Exterior lights}{set}
    % TODO: Or target fuel flow in very high DA.
    \checkitem{Mixtures}{full rich}
    \checkitem{Props}{full forward}
    \checkitem{Throttles}{39$''$ MP}
\end{checklist}

\begin{checklist}{Climb}
    % Approximately 75% power.
    \checkitem{Throttles}{31.5$''$ MP}
    \checkitem{Props}{2450 RPM}
    \checkitem{Fuel pumps}{lo (hot weather)}
    \checkitem{Airspeed}{102 KIAS}
\end{checklist}

\begin{checklist}{Cruise}
    \checkitem{Throttles}{30$''$ MP}
    \checkitem{Props}{2300 RPM}
    \checkitem{Mixtures}{10.3 GPH}
    \checkitem{Airspeed}{142 KIAS}
\end{checklist}

\begin{checklist}{Before Approach}
    \checkitem{NOTAMS}{brief}
    \checkitem{ATIS, arrival, \& approach}{brief}
    \checkitem{Terrain \& taxi}{brief}
    \checkitem{Specials}{brief}
\end{checklist}

\begin{checklist}{Approach}
    \checkitem{Altimeter}{verify}
    \checkitem{DA or MDA}{verify MSL}
    % TODO: Or target fuel flow in very high DA.
    \checkitem{Throttles}{17$''$ MP}
    \checkitem{Props}{2300 RPM}
    \checkitem{Mixtures}{full rich}
    \checkitem{Landing gear}{down}
    \checkitem{Airspeed}{100 KIAS}
\end{checklist}

\begin{checklist}{After Landing}
    \checkitem{Flaps}{retract}
    \checkitem{Alternate air}{off}
    \checkitem{Cowl flaps}{open}
\end{checklist}

\begin{checklist}{Engine Shutdown}
    \checkitem{TITs}{< 720\textdegree{F}}
    \checkitem{Avionics master}{off}
    \checkitem{Lights}{off}
    \checkitem{Throttles}{1000 RPM}
    \checkitem{Mixtures}{idle cut-off}
    \checkitem{Ignition}{off}
    \checkitem{Master switch}{off}
\end{checklist}

\begin{checklist}{V-Speeds}
    \checkitem{$V_{R}$}{66-71 KIAS}
    \checkitem{$V_{X}$}{76 KIAS}
    \checkitem{$V_{Y}$}{89 KIAS}
    \checkitem{$V_{CC}$}{102 KIAS}
    \checkitem{$V_{LE}$}{129 KIAS}
    \checkitem{$V_{LO(RET)}$}{107 KIAS}
    \checkitem{$V_{Ref}$ (Normal)}{83 KIAS}
    \checkitem{$V_{Ref}$ (Short Field)}{78 KIAS}
    \checkitem{$V_{A}$}{121-136 KIAS}
    \checkitem{$V_{MCA}$}{66 KIAS}
\end{checklist}

\onecolumn

\newtcolorbox{checklist_emerg}[1]{
    enhanced,
    title style={
        pattern={Lines[angle=60,distance=32pt,line width=16pt]},
        pattern color=black!75,
    },
    colback=white,
    colframe=black,
    fonttitle=\centering\bfseries,
    adjusted title={#1},
    sharpish corners,
    phantom=\phantomsection,
    add to list={toc}{subsection}
}

\phantomsection
\addcontentsline{toc}{section}{Emergency Procedures}
\renewcommand{\rightmark}{Emergency Procedures}

\twocolumn

\begin{checklist_emerg}{Engine Failure During Takeoff (Below 85 KIAS or Gear Down)}
    \begin{flushright}
        \checkitem{Throttles}{immediately close}
        \checkitem{Brakes}{as required / stop straight ahead}
    \end{flushright}

    \begin{center}
        \textbf{If insufficient runway remains for a complete stop:}
    \end{center}

    \checkitem{Throttles}{immediately close}
    \checkitem{Mixtures}{idle cut-off}
    \checkitem{Fuel selectors}{off}
    \checkitem{Magnetos}{off}
    \checkitem{Fuel pumps}{off}
    \checkitem{Battery master switch}{off}
    \checkitem{Brakes}{apply max. braking}

    \begin{center}
        \textbf{Maintain directional control, maneuvering to avoid obstacles if necessary.}
    \end{center}
\end{checklist_emerg}

\begin{checklist_emerg}{Engine Failure During Takeoff (Above 85 KIAS)}
    \begin{flushright}
        \checkitem{Airspeed}{maintain 89 KIAS}
        \checkitem{Mixtures}{full forward}
        \checkitem{Props}{full forward}
        \checkitem{Throttles}{39$''$ MP}
        \checkitem{Flaps}{up}
        \checkitem{Landing gear}{up}
        \checkitem{Identify}{dead foot, dead engine}
        \checkitem{Verify}{by closing throttle}
        \checkitem{Prop (inop. engine)}{feather}
        \checkitem{Climb speed}{89 KIAS}
        \checkitem{Trim}{5\textdegree{} bank toward op. engine, approx. $\frac{1}{2}$ ball slip}
        \checkitem{Cowl flap (inop. engine)}{close}
    \end{flushright}
    
    \begin{center}
        \textbf{When a positive rate of climb has been established:}
    \end{center}
    
    \checkitem{Engine securing}{complete}
\end{checklist_emerg}

\begin{checklist_emerg}{Engine Troubleshooting}
    \begin{center}
        \textbf{To attempt to restore power prior to feathering:}
    \end{center}

    \begin{flushright}
        \checkitem{Mixtures}{as required}
        \checkitem{Fuel selector}{crossfeed}
        \checkitem{Magnetos}{L or R only}
        \checkitem{Alternate air}{on}
        \checkitem{Fuel pump}{unlatch, on hi, if\\power is not immediately\\restored, off}
    \end{flushright}

    \begin{center}
        \textbf{Feather before RPM drops below 800.}
    \end{center}
\end{checklist_emerg}

\begin{checklist_emerg}{Engine Securing Procedure (Feathering Procedure)}
    \checkitem{Throttle}{close}
    \checkitem{Prop}{feather (800 RPM min.)}
    \checkitem{Mixture}{idle cut-off}
    \checkitem{Cowl flap}{close}
    \checkitem{Fuel selector}{off}
    \checkitem{Alternator}{off}
    \checkitem{Fuel pump}{off}
    \checkitem{Magnetos}{off}
    \checkitem{Electrical load}{reduce}
    \checkitem{Crossfeed}{as required}
\end{checklist_emerg}

\begin{checklist_emerg}{Air Starting Procedure (Unfeathering Procedure)}
    \checkitem{Fuel selector}{on}
    \checkitem{Fuel pump}{off}
    \checkitem{Throttle}{open $\frac{1}{4}''$}
    \checkitem{Prop}{mid range}
    \checkitem{Mixture}{full rich}
    \checkitem{Magnetos}{on}
    \checkitem{Starter}{engage}
    \checkitem{Throttle}{reduce power}
    \checkitem{Alternator (after restart)}{on}
\end{checklist_emerg}

\begin{checklist_emerg}{Manual Extension of Landing Gear}
    \begin{center}
        \textbf{Check following before extending gear manually:}
    \end{center}

    \checkitem{Circuit breakers}{check}
    \checkitem{Master switch}{on}
    \checkitem{Alternators}{check}
    \checkitem{Nav. lights}{off (daytime)}

    \begin{center}
        \textbf{To extend, reposition guard clip downward clear of knob and proceed as follows:}
    \end{center}

    \checkitem{Airspeed}{reduce (85 KIAS max.)}
    \checkitem{Landing gear}{down}
    \checkitem{Emerg. gear extend knob}{pull}
    \checkitem{Indicator lights}{3 green}

    \begin{center}
        \textbf{Leave emergency gear extension knob out.}
    \end{center}
\end{checklist_emerg}

\begin{checklist_emerg}{Electrical Fire}
    \checkitem{Master switch}{off}
    \checkitem{Avionics master}{off}
    \checkitem{Electrical switches}{off}
    \begin{center}
        \textbf{If no smoke:}
    \end{center}
    \checkitem{Circuit breakers}{note tripped}
    \checkitem{Circuit breakers}{off}
    \checkitem{Master switch}{on}
    \begin{center}
        \textbf{If no smoke:}
    \end{center}
    \checkitem{Avionics master}{on}
\end{checklist_emerg}

\begin{checklist_emerg}{Engine Fire During Start}
    \begin{center}
        \textbf{If engine has not started:}
    \end{center}

    \checkitem{Fuel selector}{off}
    \checkitem{Mixture}{idle cut-off}
    \checkitem{Throttle}{full open}
    \checkitem{Starter}{continue to crank}

    \begin{center}
        \textbf{If engine has already started and is running, continue operating to try pulling the fire into the engine.}
    \end{center}

    \begin{center}
        \textbf{If fire continues:}
    \end{center}

    \checkitem{Fuel selector}{off}
    \checkitem{Fuel pump}{off}
    \checkitem{Mixture}{idle cut-off}
    \checkitem{Throttle}{full open}
    \checkitem{Airplane}{evacuate}
\end{checklist_emerg}

\begin{checklist_emerg}{Engine Fire In Flight}
    \checkitem{Fuel selector}{off}
    \checkitem{Throttle}{close}
    \checkitem{Prop}{feather}
    \checkitem{Mixture}{idle cut-off}
    \checkitem{Heater \& defroster}{off}
    \checkitem{Cowl flap}{open}
    \checkitem{Engine securing}{complete}
\end{checklist_emerg}

\begin{checklist_emerg}{Emergency Descent}
    \begin{flushright}
        \checkitem{Throttles}{close}
        \checkitem{Props}{full forward}
        \checkitem{Mixture}{as required for\\smooth operation}
        \checkitem{Landing gear}{down}
        \checkitem{Airspeed}{129 KIAS}
    \end{flushright}
\end{checklist_emerg}

\begin{checklist_emerg}{IFR Comms Failure}
    \checkitem{Squwak}{7600}

    \begin{center}
        \textbf{If VFR, land as soon as practicable.}
    \end{center}

    \textbf{Route}
    \begin{itemize}[noitemsep]
        \item ATC assigned route
        \item Vector clearance
        \item Expected route
        \item Filed route
    \end{itemize}

    \textbf{Altitude (highest of)}
    \begin{itemize}[noitemsep]
        \item Minimum altitude for IFR
        \item Expected altitude
        \item Assigned altitude
    \end{itemize}
\end{checklist_emerg}

\phantomsection
\addcontentsline{toc}{section}{Tables and Figures}
\renewcommand{\rightmark}{Tables and Figures}


\phantomsection
\addcontentsline{toc}{subsection}{Rate of climb/descent (ft. per min)}

% https://aviation.stackexchange.com/questions/26322/how-does-the-faa-compute-the-values-in-tpps-climb-descent-table
\newcommand{\fpmatangle}[2]{\fpeval{round((#1/60) * tand(#2) * 6076.1)}}
\newcommand{\fpmatanglerounded}[2]{\fpeval{round(\fpmatangle{#1}{#2} / 5) * 5}}

\begin{table}[H]
    \caption{Rate of climb/descent (ft. per min)}

    \begin{center}
        \begin{tabular}{ccccccc}
            \toprule
            \textbf{ft/NM} & \multicolumn{5}{c}{\textbf{Ground speed (knots)}} & \textbf{Angle}
            \\\cmidrule(lr){2-6}
                           & 60                                                & 75                          & 90                          & 105                          & 120                          &                  \\
            \midrule
            210            & \fpmatanglerounded{60}{2.0}                       & \fpmatanglerounded{75}{2.0} & \fpmatanglerounded{90}{2.0} & \fpmatanglerounded{105}{2.0} & \fpmatanglerounded{120}{2.0} & 2.0\textdegree{} \\
            318            & \fpmatangle{60}{3.0}                              & \fpmatangle{75}{3.0}        & \fpmatangle{90}{3.0}        & \fpmatangle{105}{3.0}        & \fpmatangle{120}{3.0}        & 3.0\textdegree{} \\
            530            & \fpmatanglerounded{60}{5.0}                       & \fpmatanglerounded{75}{5.0} & \fpmatanglerounded{90}{5.0} & \fpmatanglerounded{105}{5.0} & \fpmatanglerounded{120}{5.0} & 5.0\textdegree{} \\
            745            & \fpmatanglerounded{60}{7.0}                       & \fpmatanglerounded{75}{7.0} & \fpmatanglerounded{90}{7.0} & \fpmatanglerounded{105}{7.0} & \fpmatanglerounded{120}{7.0} & 7.0\textdegree{} \\
            \bottomrule
        \end{tabular}
    \end{center}
\end{table}

\phantomsection
\addcontentsline{toc}{subsection}{Additional runway length required to clear low, close-in obstacles}

\begin{table}[H]
    \hypertarget{runway-length-table}{\caption{Additional runway length required to clear low, close-in obstacle}}

    \begin{center}
        \begin{tabular}{lccc}
            \toprule
                                   & \multicolumn{3}{c}{\textbf{Climb Angle}}
            \\\cmidrule(lr){2-4}
                                   & 745'/NM                                  & 530'/NM                                 & 318'/NM                                 \\
            \midrule
            % We subtract 50' from the obstacle height because we assume
            % takeoff performance data clears an initial 50' obstacle.
            \textbf{200' obstacle} & \num{\fpeval{ceil((6076*150)/745, 0)}}'  & \num{\fpeval{ceil((6076*150)/530, 0)}}' & \num{\fpeval{ceil((6076*150)/318, 0)}}' \\
            \textbf{150' obstacle} & \num{\fpeval{ceil((6076*100)/745, 0)}}'  & \num{\fpeval{ceil((6076*100)/530, 0)}}' & \num{\fpeval{ceil((6076*100)/318, 0)}}' \\
            \textbf{100' obstacle} & \num{\fpeval{ceil((6076*50)/745, 0)}}'   & \num{\fpeval{ceil((6076*50)/530, 0)}}'  & \num{\fpeval{ceil((6076*50)/318, 0)}}'  \\
            \bottomrule
        \end{tabular}
    \end{center}

    \textbf{Note:}
    \begin{itemize}
        \item Assumes takeoff performance data is based on clearing a 50' obstacle.
        \item Subtract obstacle's distance from runway end from required runway length.
        \item \hyperlink{departure-briefing}{Return back to the departure briefing.}
    \end{itemize}
\end{table}

\phantomsection
\addcontentsline{toc}{subsection}{Speed versus pivotal altitude at 100' MSL elevation}

\begin{table}[H]
    \caption{Speed versus pivotal altitude at 100' MSL elevation}

    \begin{center}
        \begin{tabular}{cc}
            \toprule
            \textbf{Ground speed (knots)} & \textbf{Approximate pivotal pltitude (MSL)}         \\
            \midrule
            % Add 100' to represent the base elevation, calculate
            % approximate pivotal altitude, then round to the nearest
            % 50.
            80                            & \num{\fpeval{round((100 + 80^2/11.3) / 50) * 50}}'  \\
            85                            & \num{\fpeval{round((100 + 85^2/11.3) / 50) * 50}}'  \\
            90                            & \num{\fpeval{round((100 + 90^2/11.3) / 50) * 50}}'  \\
            95                            & \num{\fpeval{round((100 + 95^2/11.3) / 50) * 50}}'  \\
            100                           & \num{\fpeval{round((100 + 100^2/11.3) / 50) * 50}}' \\
            110                           & \num{\fpeval{round((100 + 110^2/11.3) / 50) * 50}}' \\
            115                           & \num{\fpeval{round((100 + 115^2/11.3) / 50) * 50}}' \\
            120                           & \num{\fpeval{round((100 + 120^2/11.3) / 50) * 50}}' \\
            \bottomrule
        \end{tabular}
    \end{center}
\end{table}



\end{document}
