\phantomsection
\addcontentsline{toc}{section}{Tables and Figures}

\phantomsection
\addcontentsline{toc}{subsection}{Rate of climb/descent (ft. per min)}

% https://aviation.stackexchange.com/questions/26322/how-does-the-faa-compute-the-values-in-tpps-climb-descent-table
\newcommand{\fpmatangle}[2]{\fpeval{round((#1/60) * tand(#2) * 6076.1)}}
\newcommand{\fpmatanglerounded}[2]{\fpeval{round(\fpmatangle{#1}{#2} / 5) * 5}}

\begin{table}[H]
    \caption{Rate of climb/descent (ft. per min)}

    \begin{center}
        \begin{tabular}{ccccccc}
            \toprule
            \textbf{ft/NM} & \multicolumn{5}{c}{\textbf{Ground speed (knots)}} & \textbf{Angle}
            \\\cmidrule(lr){2-6}
                           & 60                                                & 75                          & 90                          & 105                          & 120                          &                  \\
            \midrule
            210            & \fpmatanglerounded{60}{2.0}                       & \fpmatanglerounded{75}{2.0} & \fpmatanglerounded{90}{2.0} & \fpmatanglerounded{105}{2.0} & \fpmatanglerounded{120}{2.0} & 2.0\textdegree{} \\
            318            & \fpmatangle{60}{3.0}                              & \fpmatangle{75}{3.0}        & \fpmatangle{90}{3.0}        & \fpmatangle{105}{3.0}        & \fpmatangle{120}{3.0}        & 3.0\textdegree{} \\
            530            & \fpmatanglerounded{60}{5.0}                       & \fpmatanglerounded{75}{5.0} & \fpmatanglerounded{90}{5.0} & \fpmatanglerounded{105}{5.0} & \fpmatanglerounded{120}{5.0} & 5.0\textdegree{} \\
            745            & \fpmatanglerounded{60}{7.0}                       & \fpmatanglerounded{75}{7.0} & \fpmatanglerounded{90}{7.0} & \fpmatanglerounded{105}{7.0} & \fpmatanglerounded{120}{7.0} & 7.0\textdegree{} \\
            \bottomrule
        \end{tabular}
    \end{center}
\end{table}

\phantomsection
\addcontentsline{toc}{subsection}{Additional runway length required to clear low, close-in obstacles}

\begin{table}[H]
    \hypertarget{runway-length-table}{\caption{Additional runway length required to clear low, close-in obstacle}}

    \begin{center}
        \begin{tabular}{lccc}
            \toprule
                                   & \multicolumn{3}{c}{\textbf{Climb Angle}}
            \\\cmidrule(lr){2-4}
                                   & 745'/NM                                  & 530'/NM                                 & 318'/NM                                 \\
            \midrule
            % We subtract 50' from the obstacle height because we assume
            % takeoff performance data clears an initial 50' obstacle.
            \textbf{200' obstacle} & \num{\fpeval{ceil((6076*150)/745, 0)}}'  & \num{\fpeval{ceil((6076*150)/530, 0)}}' & \num{\fpeval{ceil((6076*150)/318, 0)}}' \\
            \textbf{150' obstacle} & \num{\fpeval{ceil((6076*100)/745, 0)}}'  & \num{\fpeval{ceil((6076*100)/530, 0)}}' & \num{\fpeval{ceil((6076*100)/318, 0)}}' \\
            \textbf{100' obstacle} & \num{\fpeval{ceil((6076*50)/745, 0)}}'   & \num{\fpeval{ceil((6076*50)/530, 0)}}'  & \num{\fpeval{ceil((6076*50)/318, 0)}}'  \\
            \bottomrule
        \end{tabular}
    \end{center}

    \textbf{Note:}
    \begin{itemize}
        \item Assumes takeoff performance data is based on clearing a 50' obstacle.
        \item Subtract obstacle's distance from runway end from required runway length.
        \item \hyperlink{departure-briefing}{Return back to the departure briefing.}
    \end{itemize}
\end{table}


\newcommand{\bookemptyweight}{1634}
\newcommand{\maxgrossweight}{2550}
% Approximately two 160 lb adults, 10 lb of baggage, and 40 gal of fuel remaining.
\newcommand{\grossweight}{2150}
% The calibrated airspeeds for Vx, Vbg, and Vmd–as well as stall speed Vs and
% minimum descent rate (h_md) are porportional to gross aircraft weight.
\newcommand{\scaledvspeed}[1]{\fpeval{#1 + #1 / (2 * \grossweight) * (\grossweight - \maxgrossweight)}}
% Banking transforms these V speeds as well.
\newcommand{\bankedvspeed}[2]{\fpeval{#1 / sqrt(cosd(#2))}}
% Piper's AFM states that "linear interpolation may be used for intermediate
% gross weights."
\newcommand{\maneuveringspeed}{\fpeval{89 + (\grossweight - \bookemptyweight) * ((113 - 89) / (\maxgrossweight - \bookemptyweight))}}
% Keep in mind that there may be some error here–the indicated airspeeds for the
% V speeds mentioned above vary with air density, too (slightly).
\newcommand{\bestglidespeed}{\scaledvspeed{76}}

\phantomsection
\addcontentsline{toc}{subsection}{Archer flight maneuver entry speeds at \num{\grossweight} lbf}

\begin{table}[H]
    \caption{Archer flight maneuver entry speeds at \num{\grossweight} lbf}

    \begin{center}
        \begin{tabular}{lc}
            \toprule
            \textbf{Maneuver} & \textbf{KIAS}                                                 \\
            \midrule
            % We round maneuvering speed down to the nearest five knots because
            % the ACS says we need "an airspeed not to exceed Va."
            Steep Turns       & \fpeval{floor(\maneuveringspeed / 5) * 5}                     \\
            % The Airplane Flying Handbook says to establish "gliding speed"–not
            % any particular "best glide speed." According to Dr. John Lowry's
            % book, Performance of Light Aircraft, "for most (not all) intents
            % and purposes, banking to angle Φ is tantamount to increasing gross
            % weight from W to W / cos Φ." Essentially, banking transforms Vx,
            % Vbg, and Vmd precisely as it does stall speed. In an analysis I
            % performed for the Dakota, I discovered several interesting
            % relationships:
            %   1) Wings-level best glide speed, calculated for aircraft gross
            %      weight, is only 1-2 knots above the stall speed at 45° of
            %      bank.
            %   2) The max gross weight, wings-level best glide speed (e.g.,
            %      what's in the POH) gives an approximate 10 knot margin
            %      against the banked stall speed if flown roughly 400 lbf below
            %      max gross weight.
            %   3) Flying a steep spiral at the max gross weight, wings-level
            %      best glide speed doesn't make sense, even despite the margin.
            %      In a 50° bank, best glide speed increases too. In the Archer,
            %      at 2,150 lbf and 45° of bank, best glide speed is
            %      approximately 86 KCAS. It makes sense to be ahead of L/D max,
            %      so we round up to the nearest 5 knots:
            Steep Spiral      & \fpeval{(ceil(\bankedvspeed{\bestglidespeed}{50} / 10) * 10)} \\
            Chandelles        & \fpeval{floor(\maneuveringspeed / 5) * 5}                     \\
            Lazy Eights       & \fpeval{floor(\maneuveringspeed / 5) * 5}                     \\
            Eights on Pylons  & \fpeval{floor(\maneuveringspeed / 5) * 5}                     \\
            \bottomrule
            % TODO: Add accelerated stall entry speed.
        \end{tabular}
    \end{center}

    \textbf{Note:}
    \begin{itemize}
        \item Design maneuvering speed ($V_A$) at \num{\grossweight} lbf gross weight is $\approx$ \num{\fpeval{floor(\maneuveringspeed, 1)}} KIAS.
        \item Wings-level best glide speed ($V_{bg}$) at \num{\grossweight} lbf gross weight is $\approx$ \num{\fpeval{ceil(\bestglidespeed, 1)}} KIAS.
    \end{itemize}
\end{table}

\phantomsection
\addcontentsline{toc}{subsection}{Speed versus pivotal altitude at 100' MSL elevation}

\begin{table}[H]
    \caption{Speed versus pivotal altitude at 100' MSL elevation}

    \begin{center}
        \begin{tabular}{cc}
            \toprule
            \textbf{Ground speed (knots)} & \textbf{Approximate pivotal pltitude (MSL)}         \\
            \midrule
            % Add 100' to represent the base elevation, calculate
            % approximate pivotal altitude, then round to the nearest
            % 50.
            80                            & \num{\fpeval{round((100 + 80^2/11.3) / 50) * 50}}'  \\
            85                            & \num{\fpeval{round((100 + 85^2/11.3) / 50) * 50}}'  \\
            90                            & \num{\fpeval{round((100 + 90^2/11.3) / 50) * 50}}'  \\
            95                            & \num{\fpeval{round((100 + 95^2/11.3) / 50) * 50}}'  \\
            100                           & \num{\fpeval{round((100 + 100^2/11.3) / 50) * 50}}' \\
            110                           & \num{\fpeval{round((100 + 110^2/11.3) / 50) * 50}}' \\
            115                           & \num{\fpeval{round((100 + 115^2/11.3) / 50) * 50}}' \\
            120                           & \num{\fpeval{round((100 + 120^2/11.3) / 50) * 50}}' \\
            \bottomrule
        \end{tabular}
    \end{center}
\end{table}

