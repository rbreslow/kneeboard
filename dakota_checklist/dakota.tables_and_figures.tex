\phantomsection
\addcontentsline{toc}{section}{Tables and Figures}

\phantomsection
\addcontentsline{toc}{subsection}{Rate of climb/descent (ft. per min)}

\begin{table}[H]
    \caption{Rate of climb/descent (ft. per min)}

    \begin{center}
        \begin{tabular}{cccc}
            \toprule
            \textbf{Angle}   & \textbf{ft/NM} & \multicolumn{2}{c}{\textbf{Ground speed (knots)}}
            \\\cmidrule(lr){3-4}
                             &                & 60                                                & 90   \\
            \midrule
            3.0\textdegree{} & 318            & 318                                               & 478  \\
            5.0\textdegree{} & 530            & 530                                               & 795  \\
            7.0\textdegree{} & 745            & 745                                               & 1120 \\
            \bottomrule
        \end{tabular}
    \end{center}
\end{table}

\phantomsection
\addcontentsline{toc}{subsection}{Additional runway length required to clear low, close-in obstacles}

\begin{table}[H]
    \hypertarget{runway-length-table}{\caption{Additional runway length required to clear low, close-in obstacle}}

    \begin{center}
        \begin{tabular}{lccc}
            \toprule
                                   & \multicolumn{1}{c}{\textbf{745'/NM (7.0\textdegree{})}} & \multicolumn{1}{c}{\textbf{530'/NM (5.0\textdegree{})}} & \multicolumn{1}{c}{\textbf{318'/NM (3.0\textdegree{})}}
            \\\cmidrule(lr){2-2}\cmidrule(lr){3-3}\cmidrule(lr){4-4}
                                   & \textbf{Increase by}                                    & \textbf{Increase by}                                    & \textbf{Increase by}                                    \\\midrule
            % We subtract 50' from the obstacle height because we assume
            % takeoff performance data clears an initial 50' obstacle.
            \textbf{200' obstacle} & \num{\fpeval{ceil((6076*150)/745, 0)}}'                 & \num{\fpeval{ceil((6076*150)/530, 0)}}'                 & \num{\fpeval{ceil((6076*150)/318, 0)}}'                 \\
            \textbf{150' obstacle} & \num{\fpeval{ceil((6076*100)/745, 0)}}'                 & \num{\fpeval{ceil((6076*100)/530, 0)}}'                 & \num{\fpeval{ceil((6076*100)/318, 0)}}'                 \\
            \textbf{100' obstacle} & \num{\fpeval{ceil((6076*50)/745, 0)}}'                  & \num{\fpeval{ceil((6076*50)/530, 0)}}'                  & \num{\fpeval{ceil((6076*50)/318, 0)}}'                  \\
            \bottomrule
        \end{tabular}
    \end{center}

    \textbf{Note:}
    \begin{itemize}
        \item Assumes takeoff performance data is based on clearing a 50' obstacle.
        \item Subtract obstacle's distance from runway end from required runway length.
        \item \hyperlink{departure-briefing}{Return back to the departure briefing.}
    \end{itemize}
\end{table}

\phantomsection
\addcontentsline{toc}{subsection}{Dakota flight maneuver entry speeds}

\begin{table}[H]
    \caption{Dakota flight maneuver entry speeds}

    \begin{center}
        \begin{tabular}{lc}
            \toprule
            \textbf{Maneuver} & \textbf{KIAS} \\
            \midrule
            Steep Turns       & 110           \\
            Steep Spiral      & 85            \\
            Chandelles        & 110           \\
            Lazy Eights       & 110           \\
            Eights on Pylons  & 110           \\
            \bottomrule
            % TODO: Add accelerated stall entry speed.
        \end{tabular}
    \end{center}

    % TODO: Maneuvering speed table.
    \textbf{Note:} Maneuvering speed at 2,500 lbf aircraft gross weight is 111.8 KIAS.
\end{table}

\phantomsection
\addcontentsline{toc}{subsection}{Speed versus pivotal altitude at 100' MSL elevation}

\begin{table}[H]
    \caption{Speed versus pivotal altitude at 100' MSL elevation}

    \begin{center}
        \begin{tabular}{cc}
            \toprule
            \textbf{Ground speed (knots)} & \textbf{Approximate pivotal pltitude (MSL)}         \\
            \midrule
            % Add 100' to represent the base elevation, calculate
            % approximate pivotal altitude, then round to the nearest
            % 50.
            80                            & \num{\fpeval{round((100 + 80^2/11.3) / 50) * 50}}'  \\
            85                            & \num{\fpeval{round((100 + 85^2/11.3) / 50) * 50}}'  \\
            90                            & \num{\fpeval{round((100 + 90^2/11.3) / 50) * 50}}'  \\
            95                            & \num{\fpeval{round((100 + 95^2/11.3) / 50) * 50}}'  \\
            100                           & \num{\fpeval{round((100 + 100^2/11.3) / 50) * 50}}' \\
            110                           & \num{\fpeval{round((100 + 110^2/11.3) / 50) * 50}}' \\
            115                           & \num{\fpeval{round((100 + 115^2/11.3) / 50) * 50}}' \\
            120                           & \num{\fpeval{round((100 + 120^2/11.3) / 50) * 50}}' \\
            \bottomrule
        \end{tabular}

        \hfill

        The formula for approximate pivotal altitude is: $GS^2 / 11.3 + \textrm{MSL altitude of ground reference}$.
    \end{center}
\end{table}
