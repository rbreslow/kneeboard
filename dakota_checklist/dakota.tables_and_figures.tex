\phantomsection
\addcontentsline{toc}{section}{Tables and Figures}

\phantomsection
\addcontentsline{toc}{subsection}{Rate of climb/descent (ft. per min)}

\begin{table}[ht]
    \caption{Rate of climb/descent (ft. per min)}

    \begin{center}
        \begin{tabular}{cccc}
            \toprule
                             &       & \multicolumn{2}{c}{Ground speed (knots)}
            \\\cmidrule(lr){3-4}
            Angle            & ft/NM & 60                                       & 90   \\
            \midrule
            3.0\textdegree{} & 318   & 318                                      & 478  \\
            5.0\textdegree{} & 530   & 530                                      & 795  \\
            7.0\textdegree{} & 745   & 745                                      & 1120 \\
        \end{tabular}
    \end{center}
\end{table}

\phantomsection
\addcontentsline{toc}{subsection}{Additional runway length required to clear low, close-in obstacles}

\begin{table}[ht]
    \hypertarget{runway-length-table}{\caption{Additional runway length required to clear low, close-in obstacle}}

    \begin{center}
        \begin{tabular}{lccc}
            \toprule
                          & \multicolumn{1}{c}{745'/NM (7.0\textdegree{})} & \multicolumn{1}{c}{530'/NM (5.0\textdegree{})} & \multicolumn{1}{c}{318'/NM (3.0\textdegree{})}
            \\\cmidrule(lr){2-2}\cmidrule(lr){3-3}\cmidrule(lr){4-4}
                          & Increase by                                    & Increase by                                    & Increase by                                    \\\midrule
            % We subtract 50' from the obstacle height because we assume
            % takeoff performance data clears an initial 50' obstacle.
            200' obstacle & \num{\fpeval{ceil((6076*150)/745, 0)}}'        & \num{\fpeval{ceil((6076*150)/530, 0)}}'        & \num{\fpeval{ceil((6076*150)/318, 0)}}'        \\
            150' obstacle & \num{\fpeval{ceil((6076*100)/745, 0)}}'        & \num{\fpeval{ceil((6076*100)/530, 0)}}'        & \num{\fpeval{ceil((6076*100)/318, 0)}}'        \\
            100' obstacle & \num{\fpeval{ceil((6076*50)/745, 0)}}'         & \num{\fpeval{ceil((6076*50)/530, 0)}}'         & \num{\fpeval{ceil((6076*50)/318, 0)}}'         \\
        \end{tabular}
    \end{center}

    \textbf{Note:}
    \begin{itemize}
        \item Assumes takeoff performance data is based on clearing a 50' obstacle.
        \item Subtract obstacle's distance from runway end from required runway length.
        \item \hyperlink{departure-briefing}{Return back to the departure briefing.}
    \end{itemize}
\end{table}

\phantomsection
\addcontentsline{toc}{subsection}{Dakota flight maneuver entry speeds}

\begin{table}[ht]
    \caption{Dakota flight maneuver entry speeds}

    \begin{center}
        \begin{tabular}{lc}
            \toprule
            Maneuver         & KIAS \\
            \midrule
            Steep Turns      & 110  \\
            Steep Spiral     & 85   \\
            Chandelles       & 110  \\
            Lazy Eights      & 110  \\
            Eights on Pylons & 110  \\
            % TODO: Add accelerated stall entry speed.
        \end{tabular}
    \end{center}

    % TODO: Maneuvering speed table.
    \textbf{Note:} Maneuvering speed at 2,500 lbf aircraft gross weight is 111.8 KIAS.
\end{table}
