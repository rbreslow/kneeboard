\documentclass{article}
\usepackage{geometry}
    \geometry{
        a5paper,
        portrait,
        % The Jeppesen plate appears to be closer to 0.25in. I think
        % that 0.5in is looking best for checklists. Compromising to
        % accomodate longer line lengths.
        margin=0.25in,
        % headsep is the separation between header and text. footskip is
        % the separation between baseline of last line of text and
        % baseline of footer. The default is a bit larger. Setting these
        % to the ~line height pleases me, aesthetically.
        headsep=\baselineskip,
        footskip=\baselineskip,
        includehead,
        includefoot
    }
% NASA says:
% > The horizontal spacing between characters should be 25% of the
% > overall size and not less than one stroke width.
%
% The microtype documentation says:
% > Letterspaced fonts for which settings don’t exist will be spaced out
% > by the default of 0.1 em [...]
% AND
% > The amount is specified in thousandths of 1 em [...]
%
% So, we're scaling the default spacing by 25%, and then converting to
% housandths of an em (0.1 * 1000 * .25).
\usepackage[letterspace=25]{microtype}
% Used for the checklist frames.
\usepackage[many]{tcolorbox}
% For the preflight checklist square.
\usepackage{amssymb}
% For finer control over multi-column layouts.
\usepackage{multicol}
% For finer control over headers and footers.
\usepackage{fancyhdr}
% Used for degree symbol.
\usepackage{gensymb}
% Used for PDF ToC links.
\usepackage{hyperref}
% For printing the creation date of the document.
\usepackage{datetime2}
% Used for drawing patterns (e.g., the striped emergency procedure background).
\usepackage{tikz}
\usetikzlibrary{patterns,patterns.meta}
% Used for performance charts.
\usepackage{booktabs}
% Used for performing math inline.
\usepackage{xfp}
% Used for formatting numbers.
\usepackage{siunitx}
\sisetup{
    math-rm=\symup,
    detect-all,
    group-minimum-digits=4,
    group-separator={,}
}
% Improves positioning of tables and figures.
\usepackage{float}
% The following two packages all for sans-serif math.
\usepackage{sansmathfonts}
\usepackage[T1]{fontenc}
% https://tex.stackexchange.com/questions/84175/itemize-without-line-feed
\usepackage{enumitem}

% This is a macro that formats the checklist items and adds a new line.
\def\checkitem#1#2{
    #1\dotfill#2

}

% Set the default font family to sans-serif.
\renewcommand{\familydefault}{\sfdefault}

% NASA says:
% > The vertical spacing between lines should not be smaller than 25-33%
% > of the overall size of the font.
\renewcommand{\baselinestretch}{1.25}

% Configure the header and footer.
\pagestyle{fancy}
\fancyhf{}
\fancyhead[L]{Model 172S NAV III, Cessna Skyhawk SP}
\fancyhead[R]{\rightmark}
\fancyfoot[L]{v.\today}
\fancyfoot[R]{\thepage}

% We don't need numbered sections.
\setcounter{secnumdepth}{0}

\begin{document}

% Apply microtype tracking adjustments.
\lsstyle

% [...] you can say \raggedcolumns if you don’t want the bottom lines to
% be aligned. The default is \flushcolumns, so TEX will normally try to
% make both the top and bottom baselines of all columns align.
\raggedcolumns

% This is a macro to assist in the preflight checklist.
\def\todoitem#1{
    \item[$\square$] #1 \dotfill
}

\section{Preflight Checklist}

\subsection{Master Switch On}

\begin{itemize}
    \todoitem{Interior and exterior lights}
    \todoitem{Pitot heat}
\end{itemize}

\subsection{Walk Around}

\begin{itemize}
    \todoitem{Control surfaces and cables}
    \todoitem{Drain fuel tank sumps}
    \begin{itemize}
        \item[$\bullet$] Remove all water and sediment; verify proper fuel.
    \end{itemize}
    \todoitem{Propeller}
    \todoitem{Air inlets and alternator belt tension}
    \todoitem{Oil level (6-8 quarts)}
    \todoitem{No obvious oil or fuel leaks}
\end{itemize}

\subsection{Landing Gear}

\begin{itemize}
    \todoitem{Strut exposure ($\geq 4.5''$ for main, $\geq 3.25''$ for nose)} % TODO
    \todoitem{Visual inspection of tires} % TODO, PSI
    \todoitem{Visual inspection of brake blocks}
\end{itemize}

\subsection{Finishing Up}

\begin{itemize}
    \todoitem{Clean windshield}
    \todoitem{Prep cockpit}
    \todoitem{Engine times and Stratus}
\end{itemize}

% This is the box that surrounds the checklist items.
\newtcolorbox{checklist}[1]{
    colback=white,
    colframe=black,
    fonttitle=\centering\bfseries,
    adjusted title={#1},
    sharpish corners,
    phantom=\phantomsection,
    add to list={toc}{subsection}
}

\phantomsection
\addcontentsline{toc}{section}{Normal Procedures}

\twocolumn

\begin{checklist}{Before Start}
    \checkitem{Walk around}{completed}
    \checkitem{Seat belts}{fastened}
    \checkitem{PIC}{established}
    \checkitem{Passengers}{briefed}
\end{checklist}

\begin{checklist}{Engine Start}
    \checkitem{Parking brake}{set}
    \checkitem{Fuel selector}{both}
    \checkitem{Fuel shutoff}{on (push full in)}
    \checkitem{Alternate air}{off}
    \checkitem{Master switch}{on}
    \checkitem{Fuel pump}{on}
    \checkitem{Throttle}{$\frac{1}{4}''$}
    \checkitem{Mixture}{prime}
    \checkitem{Clear/starter}{engage}
    \checkitem{Throttle}{idle}
    \checkitem{Mixture}{lean for taxi}
    \checkitem{Oil pressure}{green arc}
    \checkitem{Electrical (M BATT and S BATT)}{check charge}
    \checkitem{Low volts annunciator}{check}
    \checkitem{Fuel pump}{off}
    \checkitem{Fuel pressure}{check}
\end{checklist}


\begin{checklist}{After Start}
    \checkitem{Avionics bus 1 \& 2}{on}
    \checkitem{Circuit breakers}{check}
    \checkitem{Garmin database}{check}
    \checkitem{Garmin self-test}{check}
    \tcblower
    \checkitem{ATIS \& clearance}{recieved}
\end{checklist}

\begin{checklist}{Before Taxi}
    \checkitem{Transponder}{set}
    \checkitem{COM \& NAV}{set}
    \checkitem{Initial altitude}{set}
    \checkitem{Initial heading}{set}
\end{checklist}

\begin{checklist}{Taxi}
    \checkitem{Exterior lights}{set}
    \checkitem{Brakes}{check}
    \checkitem{Heading indicator}{±5°}
    \checkitem{Attitude indicator}{check}
    \checkitem{Turn coordinator}{check}
\end{checklist}

% Mike Busch and John Deakin have made some interesting points about the
% engine run up process. We're running so rich, and at such low power on
% the ground, it's hard to identity any real problems with the ignition
% system outside of a dead spark plug, a dead mag, or severe spark plug
% fouling.
% TBD: Finish this.
% See: https://www.avweb.com/flight-safety/pelicans-perch-19putting-it-all-together/
\begin{checklist}{Engine Run-Up}
    \checkitem{Mixture}{full rich}
    \checkitem{Fuel selector}{both}
    % John Deakin says:
    % > The usual 1,700 RPM for running up most TCM engines (or 2,000
    % > RPM for most Lycomings) is NOT critical. I’ve seen pilots diddle
    % > and dawdle trying to get exactly 1,700 but all this does is heat
    % > the engine up for no good purpose. Plus or minus a couple
    % > hundred RPM won’t hurt a thing, so push it up to “about 1,700”
    % > or “about 2,000” and get on with it.
    % > [...]
    % > It is also becoming very clear that the mag check at low power
    % > (anything less than cruise power) is not very useful for
    % > catching problems; it’s nothing more than a quick check to catch
    % > major problems like severe plug fouling, a “hot mag,” or a dead
    % > plug, or cylinder. This was well-known in the big old radials,
    % > where mag checks are almost always performed at about 30″ MP,
    % > and up around 2,300 RPM (varies with model).
    \checkitem{Throttle}{1800 RPM}
    % \checkitem{JPI}{normalize}
    \checkitem{Mags}{check L \& R}
    \centering{
        (max drop 150; max $\Delta$ 50)
        \\
    }
    \checkitem{Vacuum}{check}
    \checkitem{Electrical}{check}
    \checkitem{Fuel pressure}{check}
    \checkitem{Oil pressure \& oil temp.}{check}
    \checkitem{Annunciators}{check}
    \checkitem{Throttle}{idle}
    \checkitem{Mixture}{lean for taxi}
\end{checklist}

\begin{checklist}{Before Takeoff}
    \checkitem{Flight controls}{check}
    \checkitem{Flight instruments}{check}
    \checkitem{Alternate air}{off}
    \checkitem{Flaps}{set}
    \checkitem{Trim}{set}

    % In the future, consider creating takeoff/landing cards to do the hard work
    % before the flight. Then, all we need to do is read through pre-computed
    % things. Also, consider how to do parts of the briefing while still on the
    % ramp, rather than right in front of the runway.
    \begin{center}
        \emph{\hypertarget{departure-briefing}{Departure briefing}}
    \end{center}

    % TODO: Include some prompt to think about any possibility of
    % tailwind, extreme crosswind, or extreme temperature inversion.
    \checkitem{\hyperlink{runway-length-table}{Takeoff distance}}{briefed}
    \checkitem{Terrain \& obstacles}{briefed}
    \checkitem{Takeoff minimums}{briefed}
    \checkitem{Departure procedure}{briefed}

    \begin{center}
        \emph{Abnormal operations}
    \end{center}

    % TODO: What will we do in case of a fire? Who will fly in an
    % emergency?
    \checkitem{Rejected takeoff}{briefed}
    \checkitem{Engine power loss}{briefed}
    \centering{
        (below \& above $\approx$ 600' AGL)
        \\
    }
\end{checklist}

\begin{checklist}{Takeoff}
    \checkitem{Time off}{noted}
    \checkitem{Doors \& windows}{secured}
    \checkitem{Exterior lights}{set}
    \checkitem{Mixture}{full rich or target EGT}
    \checkitem{Throttle}{full power}

    \begin{tcolorbox}[boxsep=0mm,left=0mm,right=0mm,colframe=black,colback=black,sharpish corners] 
        \color{white}
        \centering {
            \textbf{I WILL LOSE THE ENGINE,\\I WILL PUSH IMMEDIATELY!}
        }
    \end{tcolorbox}
\end{checklist}

\begin{checklist}{Before Approach}
    \checkitem{NOTAMS}{briefed}
    \checkitem{ATIS, arrival, \& approach}{briefed}
    \checkitem{Terrain \& taxi}{briefed}
    \checkitem{Specials}{briefed}
\end{checklist}

\begin{checklist}{Approach}
    \checkitem{Altimeter}{verify}
    \checkitem{DA or MDA}{verify MSL}
    \checkitem{Throttle}{1800 RPM}
    \checkitem{Airspeed}{90 KIAS}
    \checkitem{Mixture}{constant EGT}
\end{checklist}

\begin{checklist}{After Landing}
    \checkitem{Flaps}{retract}
    \checkitem{Mixture}{lean for taxi}
    \checkitem{Fuel pump}{off}
    \checkitem{Carburetor heat}{off}
\end{checklist}

\begin{checklist}{V-Speeds}
    \checkitem{$V_{BG}$}{76 KIAS}
    \checkitem{$V_R$ (flaps 0\degree{})}{53 KIAS}
    \checkitem{$V_R$ (flaps 25\degree{})}{41-49 KIAS}
    \checkitem{$V_{X}$}{64 KIAS}
    \checkitem{$V_{Y}$}{76 KIAS}
    \checkitem{$V_{CC}$}{87 KIAS}
    \checkitem{$V_{Ref}$ (flaps 40\degree{})}{66 KIAS}
    \checkitem{$V_{A}$}{89-113 KIAS}
    \checkitem{$V_{S_{0}}$/$V_{S_{1}}$}{49/55 KIAS}
\end{checklist}

\begin{checklist}{Engine Shutdown}
    \checkitem{Avionics bus 1 \& 2}{off}
    \checkitem{Lights}{off}
    \checkitem{Throttle}{1000 RPM}
    \checkitem{Mixture}{idle cut-off}
    \checkitem{Ignition}{off}
    \checkitem{Master switch}{off}
    \checkitem{Fuel selector}{left or right}
\end{checklist}

\onecolumn

\newtcolorbox{checklist_emerg}[1]{
    enhanced,
    title style={
        pattern={Lines[angle=60,distance=32pt,line width=16pt]},
        pattern color=black!75,
    },
    colback=white,
    colframe=black,
    fonttitle=\centering\bfseries,
    adjusted title={#1},
    sharpish corners,
    phantom=\phantomsection,
    add to list={toc}{subsection}
}

\phantomsection
\addcontentsline{toc}{section}{Emergency Procedures}
\renewcommand{\rightmark}{Emergency Procedures}

\twocolumn

\begin{checklist_emerg}{IFR Comms Failure (U.S.)}
    \checkitem{Squwak}{7600}

    \begin{center}
        \textbf{If VFR, land as soon as practicable.}
    \end{center}

    \textbf{Route}
    \begin{itemize}[noitemsep]
        \item ATC assigned route
        \item Vector clearance
        \item Expected route
        \item Filed route
    \end{itemize}

    \textbf{Altitude (highest of)}
    \begin{itemize}[noitemsep]
        \item Minimum altitude for IFR
        \item Expected altitude
        \item Assigned altitude
    \end{itemize}
\end{checklist_emerg}


\begin{checklist_emerg}{IFR Comms Failure (Europe)}
    \checkitem{Squwak}{7600}

    \begin{center}
        \textbf{If VFR, land as soon as practicable.}
    \end{center}

    \begin{center}
        \textbf{If IFR, maintain last assigned speed and altitude (or min. flight alt.) for a period of 7 minutes.}
    \end{center}

    \begin{center}
        \textbf{Refer to Jeppesen Airway Manual.}
    \end{center}
\end{checklist_emerg}

\onecolumn

\phantomsection
\addcontentsline{toc}{section}{Tables and Figures}
\renewcommand{\rightmark}{Tables and Figures}

\phantomsection
\addcontentsline{toc}{subsection}{Rate of climb/descent (ft. per min)}

% https://aviation.stackexchange.com/questions/26322/how-does-the-faa-compute-the-values-in-tpps-climb-descent-table
\newcommand{\fpmatangle}[2]{\fpeval{round((#1/60) * tand(#2) * 6076.1)}}
\newcommand{\fpmatanglerounded}[2]{\fpeval{round(\fpmatangle{#1}{#2} / 5) * 5}}

\begin{table}[H]
    \caption{Rate of climb/descent (ft. per min)}

    \begin{center}
        \begin{tabular}{ccccccc}
            \toprule
            \textbf{ft/NM} & \multicolumn{5}{c}{\textbf{Ground speed (knots)}} & \textbf{Angle}
            \\\cmidrule(lr){2-6}
                           & 60                                                & 75                          & 90                          & 105                          & 120                          &                  \\
            \midrule
            210            & \fpmatanglerounded{60}{2.0}                       & \fpmatanglerounded{75}{2.0} & \fpmatanglerounded{90}{2.0} & \fpmatanglerounded{105}{2.0} & \fpmatanglerounded{120}{2.0} & 2.0\textdegree{} \\
            318            & \fpmatangle{60}{3.0}                              & \fpmatangle{75}{3.0}        & \fpmatangle{90}{3.0}        & \fpmatangle{105}{3.0}        & \fpmatangle{120}{3.0}        & 3.0\textdegree{} \\
            530            & \fpmatanglerounded{60}{5.0}                       & \fpmatanglerounded{75}{5.0} & \fpmatanglerounded{90}{5.0} & \fpmatanglerounded{105}{5.0} & \fpmatanglerounded{120}{5.0} & 5.0\textdegree{} \\
            745            & \fpmatanglerounded{60}{7.0}                       & \fpmatanglerounded{75}{7.0} & \fpmatanglerounded{90}{7.0} & \fpmatanglerounded{105}{7.0} & \fpmatanglerounded{120}{7.0} & 7.0\textdegree{} \\
            \bottomrule
        \end{tabular}
    \end{center}
\end{table}

\phantomsection
\addcontentsline{toc}{subsection}{Additional runway length required to clear low, close-in obstacles}

\begin{table}[H]
    \hypertarget{runway-length-table}{\caption{Additional runway length required to clear low, close-in obstacle}}

    \begin{center}
        \begin{tabular}{lccc}
            \toprule
                                   & \multicolumn{3}{c}{\textbf{Climb Angle}}
            \\\cmidrule(lr){2-4}
                                   & 745'/NM                                  & 530'/NM                                 & 318'/NM                                 \\
            \midrule
            % We subtract 50' from the obstacle height because we assume
            % takeoff performance data clears an initial 50' obstacle.
            \textbf{200' obstacle} & \num{\fpeval{ceil((6076*150)/745, 0)}}'  & \num{\fpeval{ceil((6076*150)/530, 0)}}' & \num{\fpeval{ceil((6076*150)/318, 0)}}' \\
            \textbf{150' obstacle} & \num{\fpeval{ceil((6076*100)/745, 0)}}'  & \num{\fpeval{ceil((6076*100)/530, 0)}}' & \num{\fpeval{ceil((6076*100)/318, 0)}}' \\
            \textbf{100' obstacle} & \num{\fpeval{ceil((6076*50)/745, 0)}}'   & \num{\fpeval{ceil((6076*50)/530, 0)}}'  & \num{\fpeval{ceil((6076*50)/318, 0)}}'  \\
            \bottomrule
        \end{tabular}
    \end{center}

    \textbf{Note:}
    \begin{itemize}
        \item Assumes takeoff performance data is based on clearing a 50' obstacle.
        \item Subtract obstacle's distance from runway end from required runway length.
        \item \hyperlink{departure-briefing}{Return back to the departure briefing.}
    \end{itemize}
\end{table}


\newcommand{\bookemptyweight}{1663}
\newcommand{\maxgrossweight}{2550}
% Approximately two 160 lb adults, 10 lb of baggage, and 43 gal of fuel remaining.
\newcommand{\grossweight}{2350}
% The calibrated airspeeds for Vx, Vbg, and Vmd–as well as stall speed Vs and
% minimum descent rate (h_md) are porportional to gross aircraft weight.
\newcommand{\scaledvspeed}[1]{\fpeval{#1 + #1 / (2 * \grossweight) * (\grossweight - \maxgrossweight)}}
% Banking transforms these V speeds as well.
\newcommand{\bankedvspeed}[2]{\fpeval{#1 / sqrt(cosd(#2))}}
% Piper's AFM states that "linear interpolation may be used for intermediate
% gross weights."
\newcommand{\maneuveringspeed}{\fpeval{90 + (\grossweight - \bookemptyweight) * ((105 - 90) / (\maxgrossweight - \bookemptyweight))}}
% Keep in mind that there may be some error here–the indicated airspeeds for the
% V speeds mentioned above vary with air density, too (slightly).
\newcommand{\bestglidespeed}{\scaledvspeed{68}}

\phantomsection
\addcontentsline{toc}{subsection}{Skyhawk flight maneuver entry speeds at \num{\grossweight} lbf}

\begin{table}[H]
    \caption{Skyhawk flight maneuver entry speeds at \num{\grossweight} lbf}

    \begin{center}
        \begin{tabular}{lc}
            \toprule
            \textbf{Maneuver} & \textbf{KIAS}                                                 \\
            \midrule
            % We round maneuvering speed down to the nearest five knots because
            % the ACS says we need "an airspeed not to exceed Va."
            Steep Turns       & \fpeval{floor(\maneuveringspeed / 5) * 5}                     \\
            % The Airplane Flying Handbook says to establish "gliding speed"–not
            % any particular "best glide speed." According to Dr. John Lowry's
            % book, Performance of Light Aircraft, "for most (not all) intents
            % and purposes, banking to angle Φ is tantamount to increasing gross
            % weight from W to W / cos Φ." Essentially, banking transforms Vx,
            % Vbg, and Vmd precisely as it does stall speed. In an analysis I
            % performed for the Dakota, I discovered several interesting
            % relationships:
            %   1) Wings-level best glide speed, calculated for aircraft gross
            %      weight, is only 1-2 knots above the stall speed at 45° of
            %      bank.
            %   2) The max gross weight, wings-level best glide speed (e.g.,
            %      what's in the POH) gives an approximate 10 knot margin
            %      against the banked stall speed if flown roughly 400 lbf below
            %      max gross weight.
            %   3) Flying a steep spiral at the max gross weight, wings-level
            %      best glide speed doesn't make sense, even despite the margin.
            %      In a 50° bank, best glide speed increases too. In the Archer,
            %      at 2,150 lbf and 45° of bank, best glide speed is
            %      approximately 86 KCAS. It makes sense to be ahead of L/D max,
            %      so we round up to the nearest 5 knots:
            Steep Spiral      & \fpeval{(ceil(\bankedvspeed{\bestglidespeed}{50} / 10) * 10)} \\
            Chandelles        & \fpeval{floor(\maneuveringspeed / 5) * 5}                     \\
            Lazy Eights       & \fpeval{floor(\maneuveringspeed / 5) * 5}                     \\
            Eights on Pylons  & \fpeval{floor(\maneuveringspeed / 5) * 5}                     \\
            \bottomrule
            % TODO: Add accelerated stall entry speed.
        \end{tabular}
    \end{center}

    \textbf{Note:}
    \begin{itemize}
        \item Design maneuvering speed ($V_A$) at \num{\grossweight} lbf gross weight is $\approx$ \num{\fpeval{floor(\maneuveringspeed, 1)}} KIAS.
        \item Wings-level best glide speed ($V_{bg}$) at \num{\grossweight} lbf gross weight is $\approx$ \num{\fpeval{ceil(\bestglidespeed, 1)}} KIAS.
    \end{itemize}
\end{table}

\phantomsection
\addcontentsline{toc}{subsection}{Speed versus pivotal altitude at 100' MSL elevation}

\begin{table}[H]
    \caption{Speed versus pivotal altitude at 100' MSL elevation}

    \begin{center}
        \begin{tabular}{cc}
            \toprule
            \textbf{Ground speed (knots)} & \textbf{Approximate pivotal pltitude (MSL)}         \\
            \midrule
            % Add 100' to represent the base elevation, calculate
            % approximate pivotal altitude, then round to the nearest
            % 50.
            80                            & \num{\fpeval{round((100 + 80^2/11.3) / 50) * 50}}'  \\
            85                            & \num{\fpeval{round((100 + 85^2/11.3) / 50) * 50}}'  \\
            90                            & \num{\fpeval{round((100 + 90^2/11.3) / 50) * 50}}'  \\
            95                            & \num{\fpeval{round((100 + 95^2/11.3) / 50) * 50}}'  \\
            100                           & \num{\fpeval{round((100 + 100^2/11.3) / 50) * 50}}' \\
            110                           & \num{\fpeval{round((100 + 110^2/11.3) / 50) * 50}}' \\
            115                           & \num{\fpeval{round((100 + 115^2/11.3) / 50) * 50}}' \\
            120                           & \num{\fpeval{round((100 + 120^2/11.3) / 50) * 50}}' \\
            \bottomrule
        \end{tabular}
    \end{center}
\end{table}



\end{document}
