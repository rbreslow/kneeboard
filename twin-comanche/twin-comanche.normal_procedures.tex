% This is the box that surrounds the checklist items.
\newtcolorbox{checklist}[1]{
    colback=white,
    colframe=black,
    fonttitle=\centering\bfseries,
    adjusted title={#1},
    sharpish corners,
    phantom=\phantomsection,
    add to list={toc}{subsection}
}

\phantomsection
\addcontentsline{toc}{section}{Normal Procedures}
\renewcommand{\rightmark}{Normal Procedures}

\twocolumn

\begin{checklist}{Before Start}
    \checkitem{Walk around}{complete}
    \checkitem{Seat belts}{fasten}
    \checkitem{PIC \& passengers}{establish, brief}
\end{checklist}

\begin{checklist}{Engine Start}
    \checkitem{Parking brake}{set}
    \checkitem{Fuel selectors}{inboard main tanks}
    \checkitem{Alternate air}{off}
    \checkitem{Cowl flaps}{open}
    \checkitem{Alternators}{on}
    \checkitem{Master switch}{on}
    \checkitem{3 green, no red}{check}

    \begin{center}
        \emph{\hypertarget{both-engines}{Repeat for both engines}}
    \end{center}

    \checkitem{Mixture}{idle cut-off}
    \checkitem{Fuel pump}{on}
    \checkitem{Throttle}{$\frac{1}{4}''$}
    \checkitem{Mixture}{prime}
    \checkitem{Clear/starter}{engage}
    \checkitem{Throttle}{idle}
    \checkitem{Oil pressure}{above 30 PSI}
    \checkitem{Load meter}{check}
    \checkitem{Fuel pump}{off}
    \checkitem{Mixture}{lean for taxi}
\end{checklist}

\begin{checklist}{After Start}
    \checkitem{Avionics master}{on}
    \checkitem{Circuit breakers}{check}
    \checkitem{Garmin database}{check}
    \checkitem{Garmin self-test}{check}
    \checkitem{ATIS \& clearance}{recieve}
\end{checklist}

\begin{checklist}{Before Taxi}
    \checkitem{Transponder}{set}
    \checkitem{COM \& NAV}{set}
    \checkitem{Initial heading}{set}
    \checkitem{Exterior lights}{set}
\end{checklist}

\begin{checklist}{Taxi}
    \checkitem{Brakes}{check}
    \checkitem{Heading indicator}{±5°}
    \checkitem{Attitude indicator}{check}
    \checkitem{Turn coordinator}{check}
\end{checklist}

% Mike Busch and John Deakin have made some interesting points about the
% engine run up process. We're running so rich, and at such low power on
% the ground, it's hard to identity any real problems with the ignition
% system outside of a dead spark plug, a dead mag, or severe spark plug
% fouling.
% TBD: Finish this.
% See: https://www.avweb.com/flight-safety/pelicans-perch-19putting-it-all-together/
\begin{checklist}{Engine Run-Up}
    \checkitem{Mixtures}{full rich}
    \checkitem{Props}{full forward}
    % John Deakin says:
    % > The usual 1,700 RPM for running up most TCM engines (or 2,000
    % > RPM for most Lycomings) is NOT critical. I’ve seen pilots diddle
    % > and dawdle trying to get exactly 1,700 but all this does is heat
    % > the engine up for no good purpose. Plus or minus a couple
    % > hundred RPM won’t hurt a thing, so push it up to “about 1,700”
    % > or “about 2,000” and get on with it.
    % > [...]
    % > It is also becoming very clear that the mag check at low power
    % > (anything less than cruise power) is not very useful for
    % > catching problems; it’s nothing more than a quick check to catch
    % > major problems like severe plug fouling, a “hot mag,” or a dead
    % > plug, or cylinder. This was well-known in the big old radials,
    % > where mag checks are almost always performed at about 30″ MP,
    % > and up around 2,300 RPM (varies with model).
    \checkitem{Throttles}{1500 RPM}
    \checkitem{Prop}{check feathering}
    \checkitem{Throttles}{2000 RPM}
    \checkitem{Mags}{check L \& R}
    \centering{
        (max drop 175; max $\Delta$ 50)
        \\
    }
    \checkitem{Alternate air}{check}
    % When running up the Lance, Jim Cherry was taken aback when I
    % cycled the propellor at 1800 RPM. His reasoning was that it's
    % overkill, it doesn't need to be cycled that high. Unnessecary
    % strain on things, why do it? But, I forget the ballpark that he
    % reccomended. It may have been 1000-1500 RPM.
    \checkitem{Prop}{cycle}
    \centering{
        ($\downarrow$ rpm $\uparrow$ mp $\downarrow$ oil press.)
        \\
    }
    \checkitem{Vacuum}{4.8-5.1$''$Hg}
    \checkitem{Oil pressure \& oil temp.}{check}
    \checkitem{CHTs}{check}
    \checkitem{Load meter}{check}
    \checkitem{Alternate static}{check}
    \checkitem{Annunciator panel}{check}
    \checkitem{Throttles}{idle}
    \checkitem{Mixtures}{lean for taxi}
\end{checklist}

\begin{checklist}{Before Takeoff}
    \checkitem{Flight controls}{check}
    \checkitem{Flight instruments}{check}
    \checkitem{Fuel selectors}{inboard main tanks}
    \checkitem{Alternate air}{off}
    \checkitem{Cowl flaps}{open}
    \checkitem{Flaps}{set}
    \checkitem{Trim}{set}

    % In the future, consider creating takeoff/landing cards to do the hard work
    % before the flight. Then, all we need to do is read through pre-computed
    % things. Also, consider how to do parts of the briefing while still on the
    % ramp, rather than right in front of the runway.
    \begin{center}
        \emph{\hypertarget{departure-briefing}{Departure briefing}}
    \end{center}

    % TODO: Include some prompt to think about any possibility of
    % tailwind, extreme crosswind, or extreme temperature inversion.
    \checkitem{\hyperlink{runway-length-table}{Takeoff distance}}{brief}
    \checkitem{Terrain \& obstacles}{brief}
    \checkitem{Takeoff minimums}{brief}
    \checkitem{Departure procedure}{brief}

    \begin{center}
        \emph{Abnormal operations}
    \end{center}

    \checkitem{Rejected takeoff}{brief}
    \checkitem{Engine power loss}{brief}
\end{checklist}

\begin{checklist}{Takeoff}
    \checkitem{Time off}{note}
    \checkitem{Doors \& windows}{secure}
    \checkitem{Exterior lights}{set}
    % TODO: Or target fuel flow in very high DA.
    \checkitem{Fuel pumps}{on}
    \checkitem{Mixtures}{full rich}
    \checkitem{Props}{full forward}
    \checkitem{Throttles}{full foward}
\end{checklist}

\begin{checklist}{Climb}
    % Approximately 75% power.
    \checkitem{Throttles}{25$''$ MP}
    \checkitem{Props}{2500 RPM}
    \checkitem{Airspeed}{113 KIAS}
\end{checklist}

\begin{checklist}{Cruise}
    \checkitem{Throttles}{23$''$ MP}
    \checkitem{Props}{2300 RPM}
    \checkitem{Mixtures}{8.9 GPH}
    \checkitem{Fuel pumps}{off, stabilize}
    \checkitem{Airspeed}{148 KIAS}
\end{checklist}

\begin{checklist}{Before Approach}
    \checkitem{NOTAMS}{brief}
    \checkitem{ATIS, arrival, \& approach}{brief}
    \checkitem{Terrain \& taxi}{brief}
    \checkitem{Specials}{brief}
\end{checklist}

\begin{checklist}{Approach}
    \checkitem{Altimeter}{verify}
    \checkitem{DA or MDA}{verify MSL}
    % TODO: Or target fuel flow in very high DA.
    \checkitem{Throttles}{17$''$ MP}
    \checkitem{Props}{2300 RPM}
    \checkitem{Fuel pumps}{on}
    \checkitem{Mixtures}{full rich}
    \checkitem{Landing gear}{down}
    \checkitem{Airspeed}{104 KIAS}
\end{checklist}

\begin{checklist}{After Landing}
    \checkitem{Flaps}{retract}
    \checkitem{Mixtures}{lean for taxi}
    \checkitem{Fuel pumps}{off}
    \checkitem{Alternate air}{off}
    \checkitem{Cowl flaps}{open}
\end{checklist}

\begin{checklist}{Engine Shutdown}
    \checkitem{Avionics master}{off}
    \checkitem{Lights}{off}
    \checkitem{Throttles}{1000 RPM}
    \checkitem{Mixtures}{idle cut-off}
    \checkitem{Ignition}{off}
    \checkitem{Master switch}{off}
\end{checklist}

\begin{checklist}{V-Speeds}
    \checkitem{$V_{S_{0}}$}{60 KIAS}
    \checkitem{$V_{S_{1}}$}{66 KIAS}
    \checkitem{$V_{R}$}{78 KIAS}
    \checkitem{$V_{X}$}{78 KIAS}
    \checkitem{$V_{XSE}$}{82 KIAS}
    \checkitem{$V_{Y}$}{97 KIAS}
    \checkitem{$V_{YSE}$}{91 KIAS}
    \checkitem{$V_{CC}$}{113 KIAS}
    \checkitem{$V_{LE}$}{129 KIAS}
    \checkitem{$V_{LO(RET)}$}{107 KIAS}
    \checkitem{$V_{Ref}$}{83 KIAS}
    \checkitem{$V_{A}$}{117-141 KIAS}
    \checkitem{$V_{MCA}$}{78 KIAS}
\end{checklist}

\onecolumn
