\documentclass{article}
\usepackage{geometry}
    \geometry{
        a5paper,
        portrait,
        % The Jeppesen plate appears to be closer to 0.25in. I think
        % that 0.5in is looking best for checklists. Compromising to
        % accomodate longer line lengths.
        margin=0.25in,
        % headsep is the separation between header and text. footskip is
        % the separation between baseline of last line of text and
        % baseline of footer. The default is a bit larger. Setting these
        % to the ~line height pleases me, aesthetically.
        headsep=\baselineskip,
        footskip=\baselineskip,
        includehead,
        includefoot
    }
% NASA says:
% > The horizontal spacing between characters should be 25% of the
% > overall size and not less than one stroke width.
%
% The microtype documentation says:
% > Letterspaced fonts for which settings don’t exist will be spaced out
% > by the default of 0.1 em [...]
% AND
% > The amount is specified in thousandths of 1 em [...]
%
% So, we're scaling the default spacing by 25%, and then converting to
% housandths of an em (0.1 * 1000 * .25).
\usepackage[letterspace=25]{microtype}
% Used for the checklist frames.
\usepackage[many]{tcolorbox}
% For the preflight checklist square.
\usepackage{amssymb}
% For finer control over multi-column layouts.
\usepackage{multicol}
% For finer control over headers and footers.
\usepackage{fancyhdr}
% Used for degree symbol.
\usepackage{gensymb}
% Used for PDF ToC links.
\usepackage{hyperref}
% For printing the creation date of the document.
\usepackage{datetime2}
% Used for drawing patterns (e.g., the striped emergency procedure background).
\usepackage{tikz}
\usetikzlibrary{patterns,patterns.meta}
% Used for performance charts.
\usepackage{booktabs}
% Used for performing math inline.
\usepackage{xfp}
% Used for formatting numbers.
\usepackage{siunitx}
\sisetup{
    math-rm=\symup,
    detect-all,
    group-minimum-digits=4,
    group-separator={,}
}
% Improves positioning of tables and figures.
\usepackage{float}
% The following two packages all for sans-serif math.
\usepackage{sansmathfonts}
\usepackage[T1]{fontenc}
% https://tex.stackexchange.com/questions/84175/itemize-without-line-feed
\usepackage{enumitem}

% This is a macro that formats the checklist items and adds a new line.
\def\checkitem#1#2{
    #1\dotfill#2

}

% Set the default font family to sans-serif.
\renewcommand{\familydefault}{\sfdefault}

% NASA says:
% > The vertical spacing between lines should not be smaller than 25-33%
% > of the overall size of the font.
\renewcommand{\baselinestretch}{1.25}

% Configure the header and footer.
\pagestyle{fancy}
\fancyhf{}
\fancyhead[L]{Cessna 172P}
\fancyhead[R]{\rightmark}
\fancyfoot[L]{v.\today}
\fancyfoot[R]{\thepage}

% We don't need numbered sections.
\setcounter{secnumdepth}{0}

\begin{document}

% Apply microtype tracking adjustments.
\lsstyle

% [...] you can say \raggedcolumns if you don’t want the bottom lines to
% be aligned. The default is \flushcolumns, so TEX will normally try to
% make both the top and bottom baselines of all columns align.
\raggedcolumns

% This is a macro to assist in the preflight checklist.
\def\todoitem#1{
    \item[$\square$] #1 \dotfill
}

\section{Preflight Checklist}

\subsection{Master Switch On}

\begin{itemize}
    \todoitem{Interior and exterior lights}
    \todoitem{Stall warning horn}
    \todoitem{Pitot heat}
\end{itemize}

\subsection{Walk Around}

\begin{itemize}
    \todoitem{Control surfaces and cables}
    \todoitem{Drain fuel tank sumps}
    \begin{itemize}
        \item[$\bullet$] Remove all water and sediment; verify proper fuel.
    \end{itemize}
    \todoitem{Propeller}
    \todoitem{Air inlets and alternator belt tension}
    \todoitem{Oil level (5-7 quarts)}
    \todoitem{Pull out strainer drain knob for 4 sec, then check closed}
    \todoitem{No obvious oil or fuel leaks}
\end{itemize}

\subsection{Landing Gear}

\begin{itemize}
    \todoitem{Strut exposure}
    \todoitem{Visual inspection of tires (28 PSI for main, 34 PSI for nose)}
    \todoitem{Visual inspection of brake blocks}
\end{itemize}

\subsection{Finishing Up}

\begin{itemize}
    \todoitem{Clean windshield}
    \todoitem{Prep cockpit}
    \todoitem{Engine times and Stratus}
\end{itemize}

% This is the box that surrounds the checklist items.
\newtcolorbox{checklist}[1]{
    colback=white,
    colframe=black,
    fonttitle=\centering\bfseries,
    adjusted title={#1},
    sharpish corners,
    phantom=\phantomsection,
    add to list={toc}{subsection}
}

\phantomsection
\addcontentsline{toc}{section}{Normal Procedures}

\twocolumn

\begin{checklist}{Before Start}
    \checkitem{Walk around}{complete}
    \checkitem{Seat belts}{fasten}
    \checkitem{PIC}{establish}
    \checkitem{Passengers}{brief}
\end{checklist}

\begin{checklist}{Engine Start}
    \checkitem{Parking brake}{set}
    \checkitem{Fuel selector}{both}
    \checkitem{Avionics master}{check off}
    \checkitem{Carburetor heat}{off}
    \checkitem{Master switch}{on}
    \checkitem{Mixture}{full rich}
    \checkitem{Throttle}{$\frac{1}{4}''$}
    \checkitem{Prime}{5-6 shots, if cold}
    \checkitem{Clear/starter}{engage}
    \checkitem{Throttle}{idle}
    \checkitem{Mixture}{lean for taxi}
    \checkitem{Oil pressure}{above red radial}
    \checkitem{Load meter}{check}
\end{checklist}

\begin{checklist}{After Start}
    \checkitem{Avionics master}{on}
    \checkitem{Circuit breakers}{check}
    \checkitem{Garmin database}{check}
    \checkitem{Garmin self-test}{check}
    \tcblower
    \checkitem{ATIS \& clearance}{recieve}
\end{checklist}

\newpage

\begin{checklist}{Before Taxi}
    \checkitem{Transponder}{set}
    \checkitem{COM \& NAV}{set}
    \checkitem{Initial altitude}{set}
    \checkitem{Initial heading}{set}
\end{checklist}

\begin{checklist}{Taxi}
    \checkitem{Exterior lights}{set}
    \checkitem{Brakes}{check}
    \checkitem{Heading indicator}{±5°}
    \checkitem{Attitude indicator}{check}
    \checkitem{Turn coordinator}{check}
\end{checklist}

% Mike Busch and John Deakin have made some interesting points about the
% engine run up process. We're running so rich, and at such low power on
% the ground, it's hard to identity any real problems with the ignition
% system outside of a dead spark plug, a dead mag, or severe spark plug
% fouling.
% TBD: Finish this.
% See: https://www.avweb.com/flight-safety/pelicans-perch-19putting-it-all-together/
\begin{checklist}{Engine Run-Up}
    \checkitem{Mixture}{full rich}
    % John Deakin says:
    % > The usual 1,700 RPM for running up most TCM engines (or 2,000
    % > RPM for most Lycomings) is NOT critical. I’ve seen pilots diddle
    % > and dawdle trying to get exactly 1,700 but all this does is heat
    % > the engine up for no good purpose. Plus or minus a couple
    % > hundred RPM won’t hurt a thing, so push it up to “about 1,700”
    % > or “about 2,000” and get on with it.
    % > [...]
    % > It is also becoming very clear that the mag check at low power
    % > (anything less than cruise power) is not very useful for
    % > catching problems; it’s nothing more than a quick check to catch
    % > major problems like severe plug fouling, a “hot mag,” or a dead
    % > plug, or cylinder. This was well-known in the big old radials,
    % > where mag checks are almost always performed at about 30″ MP,
    % > and up around 2,300 RPM (varies with model).
    \checkitem{Throttle}{1700 RPM}
    % \checkitem{JPI}{normalize}
    \checkitem{Mags}{check L \& R}
    \centering{
        (max drop 125; max $\Delta$ 50)
        \\
    }
    \checkitem{Carburetor heat}{check}
    \checkitem{Load meter}{check}
    \checkitem{Oil pressure \& oil temp.}{check}
    \checkitem{Alternate static}{check}
    \checkitem{Annunciator panel}{check}
    \checkitem{Throttle}{idle}
    \checkitem{Mixture}{lean for taxi}
\end{checklist}

\begin{checklist}{Before Takeoff}
    \checkitem{Flight controls}{check}
    \checkitem{Flight instruments}{check}
    \checkitem{Carburetor heat}{off}
    \checkitem{Flaps}{set}
    \checkitem{Trim}{set}

    % In the future, consider creating takeoff/landing cards to do the hard work
    % before the flight. Then, all we need to do is read through pre-computed
    % things. Also, consider how to do parts of the briefing while still on the
    % ramp, rather than right in front of the runway.
    \begin{center}
        \emph{\hypertarget{departure-briefing}{Departure briefing}}
    \end{center}

    % TODO: Include some prompt to think about any possibility of
    % tailwind, extreme crosswind, or extreme temperature inversion.
    \checkitem{\hyperlink{runway-length-table}{Takeoff distance}}{brief}
    \checkitem{Terrain \& obstacles}{brief}
    \checkitem{Takeoff minimums}{brief}
    \checkitem{Departure procedure}{brief}

    \begin{center}
        \emph{Abnormal operations}
    \end{center}

    % TODO: What will we do in case of a fire? Who will fly in an
    % emergency?
    \checkitem{Rejected takeoff}{brief}
    \checkitem{Engine power loss}{brief}
    \centering{
        (below \& above $\approx$ 600' AGL)
        \\
    }
\end{checklist}

\begin{checklist}{Takeoff}
    \checkitem{Time off}{note}
    \checkitem{Doors \& windows}{secure}
    \checkitem{Exterior lights}{set}
    \checkitem{Mixture}{full rich or max RPM}
    \checkitem{Throttle}{full power}

    \begin{tcolorbox}[boxsep=0mm,left=0mm,right=0mm,colframe=black,colback=black,sharpish corners] 
        \color{white}
        \centering {
            \textbf{I WILL LOSE THE ENGINE,\\I WILL PUSH IMMEDIATELY!}
        }
    \end{tcolorbox}
\end{checklist}

\begin{checklist}{Climb}
    \checkitem{Mixture}{full rich or max RPM}
    \checkitem{Throttle}{full power}
    \checkitem{Airspeed}{70 - 85 KIAS}
\end{checklist}

% TODO: Do 55% power as well under similar conditions. For holding, better
% range, etc.
\begin{checklist}{Cruise}
    % From Reims F172N flight manual, cruise performance, max weight 1043 kg,
    % standard temperature, 2000 ft pressure altitude.
    \checkitem{Throttle}{2400 RPM}
    \checkitem{Mixture}{best power}
    \checkitem{Fuel flow}{7.5 GPH}
    % https://aerotoolbox.com/airspeed-conversions/ says 111 kts TAS at 2000 ft
    % is 108 kts CAS. F172N airspeed correction table says 108 kts CAS is 110
    % kts IAS.
    \checkitem{Airspeed}{110 KIAS}
    \checkitem{Cruise performance table}{check}
\end{checklist}

\begin{checklist}{Before Approach}
    \checkitem{NOTAMs}{brief}
    \checkitem{ATIS, arrival, \& approach}{brief}
    \checkitem{Terrain \& taxi}{brief}
    \checkitem{Specials}{brief}
\end{checklist}

\begin{checklist}{Approach}
    \checkitem{Altimeter}{verify}
    \checkitem{DA or MDA}{verify MSL}
    \checkitem{Throttle}{1800 RPM}
    \checkitem{Airspeed}{90 KIAS}
    \checkitem{Mixture}{full rich}
\end{checklist}

\begin{checklist}{After Landing}
    \checkitem{Flaps}{retract}
    \checkitem{Mixture}{lean for taxi}
    \checkitem{Carburetor heat}{off}
\end{checklist}

\begin{checklist}{Engine Shutdown}
    \checkitem{Avionics master}{off}
    \checkitem{Lights}{off}
    \checkitem{Throttle}{1000 RPM}
    \checkitem{Mixture}{idle cut-off}
    \checkitem{Ignition}{off}
    \checkitem{Master switch}{off}
    \checkitem{Fuel selector}{left or right}
\end{checklist}

% Numbers obtained from Cessna 172P POH at maximum weight 2400 lbs.
\begin{checklist}{V-Speeds}
    \checkitem{$V_{BG}$}{65 KIAS}
    \checkitem{$V_R$}{55 KIAS}
    \checkitem{$V_R$ (Short Field 10\degree{})}{51 KIAS}
    \checkitem{$V_X$}{60 KIAS}
    \checkitem{$V_X$ (Short Field 10\degree{})}{56 KIAS}
    \checkitem{$V_Y$}{76 KIAS}
    \checkitem{$V_{CC}$}{70-85 KIAS}
    \checkitem{$V_{Ref}$ (Normal 0\degree{})}{65-75 KIAS}
    \checkitem{$V_{Ref}$ (Normal 30\degree{})}{60-70 KIAS}
    \checkitem{$V_{Ref}$ (Short Field 30\degree{})}{61 KIAS}
    \checkitem{$V_A$}{82-99 KIAS}
    \checkitem{$V_{S_{0}}$/$V_{S_{1}}$}{33/44 KIAS}
\end{checklist}

\onecolumn

\newtcolorbox{checklist_emerg}[1]{
    enhanced,
    title style={
        pattern={Lines[angle=60,distance=32pt,line width=16pt]},
        pattern color=black!75,
    },
    colback=white,
    colframe=black,
    fonttitle=\centering\bfseries,
    adjusted title={#1},
    sharpish corners,
    phantom=\phantomsection,
    add to list={toc}{subsection}
}

\phantomsection
\addcontentsline{toc}{section}{Emergency Procedures}

\twocolumn

\begin{checklist_emerg}{Electrical Fire\\(Smoke in Cabin)}
    \checkitem{Master switch}{off}
    \checkitem{Avionics master}{off}
    \checkitem{Electrical switches}{off}
    \begin{center}
        \textbf{If no smoke:}
    \end{center}
    \checkitem{Circuit breakers}{note tripped}
    \checkitem{Circuit breakers}{off}
    \checkitem{Master switch}{on}
    \begin{center}
        \textbf{If no smoke:}
    \end{center}
    \checkitem{Avionics master}{on}
\end{checklist_emerg}

\begin{checklist_emerg}{Alternator Failure}
    \checkitem{Verify failure}{}
    \checkitem{Reduce electrical load as much as possible}{}
    \checkitem{Alt circuit breakers}{check}
    \checkitem{Alt switch}{off, wait, then on}
    \begin{center}
        \textbf{If no output:}
    \end{center}
    \checkitem{Alt switch}{off}
    \checkitem{Reduce electrical load and land as soon as practical}{}
\end{checklist_emerg}

\begin{checklist_emerg}{IFR Comms Failure (U.S.)}
    \checkitem{Squwak}{7600}

    \begin{center}
        \textbf{If VFR, land as soon as practicable.}
    \end{center}

    \textbf{Route}
    \begin{itemize}[noitemsep]
        \item ATC assigned route
        \item Vector clearance
        \item Expected route
        \item Filed route
    \end{itemize}

    \textbf{Altitude (highest of)}
    \begin{itemize}[noitemsep]
        \item Minimum altitude for IFR
        \item Expected altitude
        \item Assigned altitude
    \end{itemize}
\end{checklist_emerg}

\begin{checklist_emerg}{IFR Comms Failure (Europe)}
    \checkitem{Squwak}{7600}

    \begin{center}
        \textbf{If VFR, land as soon as practicable.}
    \end{center}

    \begin{center}
        \textbf{If IFR, maintain last assigned speed and altitude (or min. flight alt.) for a period of 7 minutes.}
    \end{center}

    \begin{center}
        \textbf{Refer to Jeppesen Airway Manual.}
    \end{center}
\end{checklist_emerg}

\textbf{Note:} Checklist is a WIP. Missing emergency procedures (like engine failure) as per 14 CFR § 91.503.

\onecolumn


\end{document}
