\documentclass{article}
\usepackage{geometry}
    \geometry{
        a5paper,
        portrait,
        % The Jeppesen plate appears to be closer to 0.25in. I think
        % that 0.5in is looking best for checklists. Compromising to
        % accomodate longer line lengths.
        margin=0.25in,
        rmargin=0.375in,
        % I decided arbitrarily on these values to maximize margins,
        % headers, and content in one page (and satisfy fancyhdr).
        % https://tex.stackexchange.com/questions/132170/what-do-headheight-headsep-etc-do-in-the-vmargin-package
        headsep=12pt,
        footskip=12pt,
        includehead,
        includefoot
    }
% NASA says:
% > The horizontal spacing between characters should be 25% of the
% > overall size and not less than one stroke width.
%
% The microtype documentation says:
% > Letterspaced fonts for which settings don’t exist will be spaced out
% > by the default of 0.1 em [...]
% AND
% > The amount is specified in thousandths of 1 em [...]
%
% So, we're scaling the default spacing by 25%, and then converting to
% housandths of an em (0.1 * 1000 * .25).
\usepackage[letterspace=25]{microtype}
% For finer control over multi-column layouts.
\usepackage{multicol}
% For finer control over headers and footers.
\usepackage{fancyhdr}
% Used for degree symbol.
\usepackage{gensymb}
% Used for PDF ToC links.
\usepackage{hyperref}
% For printing the creation date of the document.
\usepackage{datetime2}
% Used for performance charts.
\usepackage{booktabs}
% Used for performing math inline.
\usepackage{xfp}
% Used for formatting numbers.
\usepackage{siunitx}
\sisetup{
    math-rm=\symup,
    detect-all,
    group-minimum-digits=4,
    group-separator={,}
}

% Set the default font family to sans-serif.
\renewcommand{\familydefault}{\sfdefault}

% NASA says:
% > The vertical spacing between lines should not be smaller than 25-33%
% > of the overall size of the font.
\renewcommand{\baselinestretch}{1.25}

% Configure the header and footer.
\pagestyle{fancy}
\fancyhf{}
\fancyhead[L]{Cheat Sheet}
\fancyfoot[L]{v.\today}
\fancyfoot[R]{\thepage}

% We don't need numbered sections.
\setcounter{secnumdepth}{0}

\begin{document}

% Apply microtype tracking adjustments.
\lsstyle

\begin{table}[ht]
    \caption{Rate of climb/descent (ft. per min)}

    \begin{center}
        \begin{tabular}{cccc}
            \toprule
                             &       & \multicolumn{2}{c}{Ground speed (knots)}
            \\\cmidrule(lr){3-4}
            Angle            & ft/NM & 60                                       & 90   \\
            \midrule
            3.0\textdegree{} & 318   & 318                                      & 478  \\
            5.0\textdegree{} & 530   & 530                                      & 795  \\
            7.0\textdegree{} & 745   & 745                                      & 1120 \\
        \end{tabular}
    \end{center}
\end{table}

\begin{table}[ht]
    \caption{Additional runway length required to clear low, close-in obstacles}

    \begin{center}
        \begin{tabular}{lccc}
            \toprule
                          & \multicolumn{1}{c}{745'/NM (7.0\textdegree{})} & \multicolumn{1}{c}{530'/NM (5.0\textdegree{})} & \multicolumn{1}{c}{318'/NM (3.0\textdegree{})}
            \\\cmidrule(lr){2-2}\cmidrule(lr){3-3}\cmidrule(lr){4-4}
                          & Increase by                                    & Increase by                                    & Increase by                                    \\\midrule
            % We subtract 50' from the obstacle height because we assume
            % takeoff performance data clears an initial 50' obstacle.
            200' obstacle & \num{\fpeval{ceil((6076*150)/745, 0)}}'        & \num{\fpeval{ceil((6076*150)/530, 0)}}'        & \num{\fpeval{ceil((6076*150)/318, 0)}}'        \\
            150' obstacle & \num{\fpeval{ceil((6076*100)/745, 0)}}'        & \num{\fpeval{ceil((6076*100)/530, 0)}}'        & \num{\fpeval{ceil((6076*100)/318, 0)}}'        \\
            100' obstacle & \num{\fpeval{ceil((6076*50)/745, 0)}}'         & \num{\fpeval{ceil((6076*50)/530, 0)}}'         & \num{\fpeval{ceil((6076*50)/318, 0)}}'         \\
        \end{tabular}
    \end{center}

    \textbf{Note:}
    \begin{itemize}
        \item Assumes takeoff performance data is based on clearing a 50' obstacle.
        \item Subtract obstacle's distance from runway end from required runway length.
    \end{itemize}
\end{table}

\begin{table}[ht]
    \caption{Dakota flight maneuver entry speeds}

    \begin{center}
        \begin{tabular}{lc}
            \toprule
            Maneuver         & KIAS \\
            \midrule
            Steep Turns      & 110  \\
            Steep Spiral     & 85   \\
            Chandelles       & 110  \\
            Lazy Eights      & 110  \\
            Eights on Pylons & 110  \\
            % TODO: Add accelerated stall entry speed.
        \end{tabular}
    \end{center}

    % TODO: Maneuvering speed table.
    \textbf{Note:} Maneuvering speed at 2,500 lbf aircraft gross weight is 111.8 KIAS.
\end{table}

\end{document}
