\documentclass{article}
\usepackage{geometry}
    \geometry{
        a5paper,
        portrait,
        % The Jeppesen plate appears to be closer to 0.25in. I think
        % that 0.5in is looking best for checklists. Compromising to
        % accomodate longer line lengths.
        margin=0.25in,
        rmargin=0.375in,
        % I decided arbitrarily on these values to maximize margins,
        % headers, and content in one page (and satisfy fancyhdr).
        % https://tex.stackexchange.com/questions/132170/what-do-headheight-headsep-etc-do-in-the-vmargin-package
        headsep=12pt,
        footskip=12pt,
        includehead,
        includefoot
    }
% NASA says:
% > The horizontal spacing between characters should be 25% of the
% > overall size and not less than one stroke width.
%
% The microtype documentation says:
% > Letterspaced fonts for which settings don’t exist will be spaced out
% > by the default of 0.1 em [...]
% AND
% > The amount is specified in thousandths of 1 em [...]
%
% So, we're scaling the default spacing by 25%, and then converting to
% housandths of an em (0.1 * 1000 * .25).
\usepackage[letterspace=25]{microtype}
% For finer control over multi-column layouts.
\usepackage{multicol}
% For finer control over headers and footers.
\usepackage{fancyhdr}
% Used for degree symbol.
\usepackage{gensymb}
% Used for PDF ToC links.
\usepackage{hyperref}
% For printing the creation date of the document.
\usepackage{datetime2}
% Used for performance charts.
\usepackage{booktabs}

% Set the default font family to sans-serif.
\renewcommand{\familydefault}{\sfdefault}

% NASA says:
% > The vertical spacing between lines should not be smaller than 25-33%
% > of the overall size of the font.
\renewcommand{\baselinestretch}{1.25}

% Configure the header and footer.
\pagestyle{fancy}
\fancyhf{}
\fancyhead[L]{Cheat Sheet}
\fancyfoot[L]{v.\today}
\fancyfoot[R]{\thepage}

% We don't need numbered sections.
\setcounter{secnumdepth}{0}

\begin{document}

% Apply microtype tracking adjustments.
\lsstyle

% [...] you can say \raggedcolumns if you don’t want the bottom lines to
% be aligned. The default is \flushcolumns, so TEX will normally try to
% make both the top and bottom baselines of all columns align.
\raggedcolumns

\begin{figure}[ht]
    \centering

    \begin{tabular}{lccc}
        \toprule
                      & \multicolumn{1}{c}{745'/NM (7.0\textdegree{})} & \multicolumn{1}{c}{530'/NM (5.0\textdegree{})} & \multicolumn{1}{c}{318'/NM (3.0\textdegree{})}
        \\\cmidrule(lr){2-2}\cmidrule(lr){3-3}\cmidrule(lr){4-4}
                      & RTRL                                           & RTRL                                           & RTRL                                           \\\midrule
        200' Obstacle & 1,631'                                         & 1,146'                                         & 3,821'                                         \\
        150' Obstacle & 1,223'                                         & 1,719'                                         & 2,866'                                         \\
        100' Obstacle & 815'                                           & 1,146'                                         & 1,910'                                         \\
    \end{tabular}

    \hfill

    \textbf{Note:} Subtract obstacle distance from runway end from RTRL.

    \caption{Reduce Takeoff Runway Length (RTRL) by some distance to clear an obstacle of some height}
\end{figure}

\begin{figure}[ht]
    \centering

    \begin{tabular}{lc}
        \toprule
        Maneuver         & KIAS \\
        \midrule
        Steep Turns      & 110  \\
        Steep Spiral     & 85   \\
        Chandelles       & 110  \\
        Lazy Eights      & 110  \\
        Eights on Pylons & 110  \\
        % TODO: Add accelerated stall entry speed.
    \end{tabular}

    \hfill

    % TODO: Maneuvering speed table.
    \textbf{Note:} Maneuvering speed at 2,500 lbf aircraft gross weight is 111.8 KIAS.

    \caption{Dakota flight maneuver entry speeds}
\end{figure}

\end{document}
