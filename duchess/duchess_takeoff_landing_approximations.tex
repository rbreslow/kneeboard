\addcontentsline{toc}{section}{Takeoff and Landing Approximations}

\phantomsection
\addcontentsline{toc}{subsection}{Takeoff Distance Over a 50' Obstacle At 3,600 lbf}

\begin{table}[H]
    \caption{Takeoff Distance Over a 50' Obstacle At 3,600 lbf}

    \begin{center}
        \begin{tabular}{ccc}
            \toprule
            \textbf{P\textsubscript{alt}} & \textbf{OAT (\textdegree{C})} & \textbf{Distance (ft)} \\
            \midrule
            1,000' & 15 \textdegree{C}               & 2,000'          \\
            5,000' &  40 \textdegree{C}                 & 5,000'              \\
            &                  &              
        \end{tabular}
    \end{center}
\end{table}

\phantomsection
\addcontentsline{toc}{subsection}{Landing Distance Over a 50' Obstacle At 3,600 lbf}

\begin{table}[H]
    \caption{Landing Distance Over a 50' Obstacle At 3,600 lbf}
    \begin{center}
        \begin{tabular}{ccc}
            \toprule
            \textbf{P\textsubscript{alt}} & \textbf{OAT (\textdegree{C})} & \textbf{Distance (ft)} \\
            \midrule
            1,000' & 15 \textdegree{C}               & 1,970'          \\
            &                  &              
        \end{tabular}
    \end{center}
\end{table}


\subsection{Climb Notes}

\begin{itemize}
    \item{At 3,900 lbf around 5,000' D\textsubscript{alt}, you should see 1,000 fpm with AEO.}
    \item{At 3,600 lbf around 1,000' D\textsubscript{alt}, you should see $\approx$ 250 fpm with OEI.}
    \item{You're not even hitting 200'/NM. Maybe at $V_Y$ between 0-1000'. Not enough to make it to the enroute environment IFR.}
\end{itemize}
