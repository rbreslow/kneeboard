\documentclass{article}
\usepackage{geometry}
    \geometry{
        a5paper,
        portrait,
        % The Jeppesen plate appears to be closer to 0.25in. I think
        % that 0.5in is looking best for checklists. Compromising to
        % accomodate longer line lengths.
        margin=0.25in,
        rmargin=0.375in,
        % headsep is the separation between header and text. footskip is
        % the separation between baseline of last line of text and
        % baseline of footer. The default is a bit larger. Setting these
        % to the ~line height pleases me, aesthetically.
        headsep=\baselineskip,
        footskip=\baselineskip,
        includehead,
        includefoot
    }
% NASA says:
% > The horizontal spacing between characters should be 25% of the
% > overall size and not less than one stroke width.
%
% The microtype documentation says:
% > Letterspaced fonts for which settings don’t exist will be spaced out
% > by the default of 0.1 em [...]
% AND
% > The amount is specified in thousandths of 1 em [...]
%
% So, we're scaling the default spacing by 25%, and then converting to
% housandths of an em (0.1 * 1000 * .25).
\usepackage[letterspace=25]{microtype}
% Used for the checklist frames.
\usepackage[many]{tcolorbox}
% For the preflight checklist square.
\usepackage{amssymb}
% For finer control over multi-column layouts.
\usepackage{multicol}
% For finer control over headers and footers.
\usepackage{fancyhdr}
% Used for degree symbol.
\usepackage{gensymb}
% Used for PDF ToC links.
\usepackage{hyperref}
% For printing the creation date of the document.
\usepackage{datetime2}
% Used for drawing patterns (e.g., the striped emergency procedure background).
\usepackage{tikz}
\usetikzlibrary{patterns,patterns.meta}
% Used for performance charts.
\usepackage{booktabs}
% Used for performing math inline.
\usepackage{xfp}
% Used for formatting numbers.
\usepackage{siunitx}
\sisetup{
    math-rm=\symup,
    detect-all,
    group-minimum-digits=4,
    group-separator={,}
}
% Improves positioning of tables and figures.
\usepackage{float}
% The following two packages all for sans-serif math.
\usepackage{sansmathfonts}
\usepackage[T1]{fontenc}

% This is a macro that formats the checklist items and adds a new line.
\def\checkitem#1#2{
    #1\dotfill#2

}

% Set the default font family to sans-serif.
\renewcommand{\familydefault}{\sfdefault}

% NASA says:
% > The vertical spacing between lines should not be smaller than 25-33%
% > of the overall size of the font.
\renewcommand{\baselinestretch}{1.25}

% Configure the header and footer.
\pagestyle{fancy}
\fancyhf{}
\fancyhead[L]{Model 76, Beechcraft Duchess 76}
\fancyfoot[L]{v.\today}
\fancyfoot[R]{\thepage}

% We don't need numbered sections.
\setcounter{secnumdepth}{0}

\begin{document}

% Apply microtype tracking adjustments.
\lsstyle

% [...] you can say \raggedcolumns if you don’t want the bottom lines to
% be aligned. The default is \flushcolumns, so TEX will normally try to
% make both the top and bottom baselines of all columns align.
\raggedcolumns

% This is a macro to assist in the preflight checklist.
\def\todoitem#1{
    \item[$\square$] #1 \dotfill
}

\section{Preflight Checklist}

\subsection{Battery Switch On}

\begin{itemize}
    \todoitem{Interior and exterior lights}
    \todoitem{Stall warning horns}
    \todoitem{Pitot heat}
\end{itemize}

\subsection{Walk Around}


\begin{itemize}
    \todoitem{Control surfaces and cables}
    \todoitem{Drain fuel tank sumps}
    \begin{itemize}
        \item[$\bullet$] Remove all water and sediment; verify proper fuel.
    \end{itemize}
    \todoitem{Propeller}
    \todoitem{Air inlets and cowl flaps}
    \todoitem{Oil level (6-8 quarts)}
    \todoitem{No obvious oil or fuel leaks}
    \todoitem{Static ports clear of obstructions}
\end{itemize}

\subsection{Landing Gear}

\begin{itemize}
    \todoitem{Strut exposure ($\approxeq 2.0''$ for main, $\approxeq 4.25''$ for nose)}
    \todoitem{Visual inspection of tires ($\approxeq 38$ psi)}
    \todoitem{Visual inspection of brake blocks}
\end{itemize}

\subsection{Finishing Up}

\begin{itemize}
    \todoitem{Clean windshield}
    \todoitem{Prep cockpit}
    \todoitem{Hobbs/engines times and Stratus}
\end{itemize}

\phantomsection
\addcontentsline{toc}{section}{Normal Procedures}

\twocolumn

\begin{checklist}{Before Start}
    \checkitem{Walk around}{completed}
    \checkitem{Butterfly tool}{stowed}
    \checkitem{Seat belts}{fastened}
    \checkitem{PIC}{established}
    \checkitem{Passengers}{briefed}
\end{checklist}

\begin{checklist}{Engine Start}
    \checkitem{Parking brake}{set}
    \checkitem{All avionics \& lights}{off}
    \checkitem{Circuit breakers}{check}
    \checkitem{Landing gear}{down}
    \checkitem{Fuel}{check, then on}
    \checkitem{Carburetor heat}{off}
    \checkitem{Cowl flaps}{set}
    \checkitem{Battery switch}{on}
    \checkitem{Fuel indicators}{check}
    \checkitem{3-in-the-green}{check}

    \begin{center}
        \emph{\hypertarget{both-engines}{Repeat for both engines}}
    \end{center}

    \checkitem{Mixture}{full rich}
    \checkitem{Prop}{full forward}
    \checkitem{Throttle}{$\frac{1}{4}$ travel}
    \checkitem{Fuel pump}{on}
    \checkitem{Prime}{5-6 sec.}
    \checkitem{Clear/starter}{engage}
    \checkitem{Throttle}{idle}
    \checkitem{Mixture}{lean for taxi}
    \checkitem{Oil pressure}{above red radial}
    \checkitem{Alternator switch}{on}
    \checkitem{Load meter}{check charging}
    \checkitem{Fuel pump}{off}
    \checkitem{Fuel pressure}{check}
\end{checklist}

\begin{checklist}{After Start}
    \checkitem{Avionics bus 1 \& 2}{on}
    \checkitem{Circuit breakers}{check}
    \checkitem{Garmin database}{check}
    \checkitem{Garmin self-test}{check}
    \checkitem{ATIS \& clearance}{recieved}
\end{checklist}

% \begin{checklist}{Before Taxi}
%     \checkitem{Transponder}{set}
%     \checkitem{COM \& NAV}{set}
%     \checkitem{Initial altitude}{set}
%     \checkitem{Initial heading}{set}
% \end{checklist}

\begin{checklist}{Taxi}
    \checkitem{Brakes}{check}
    \checkitem{Heading indicator}{±5°}
    \checkitem{Attitude indicator}{check}
    \checkitem{Turn coordinator}{check}
\end{checklist}

\begin{checklist}{Engine Run-Up}
    \checkitem{Mixture}{full rich}
    \checkitem{Prop}{full forward}
    \checkitem{Throttle}{2200 RPM}
    \checkitem{Mags}{check L \& R}
    \centering{
        (max drop 175; max $\Delta$ 50)
        \\
    }
    \checkitem{Carburetor heat}{check}
    \checkitem{Prop}{cycle}
    \centering{
        (100-200 rpm drop)
        \\
    }
    \checkitem{Throttle}{1500 RPM}
    \checkitem{Prop}{cycle}
    \centering{
        ($\downarrow$ rpm $\uparrow$ mp $\downarrow$ oil press.)
        \\
    }
    \checkitem{Vacuum}{4.9-5.1$''$Hg}
    \checkitem{Load meter}{check}
    \checkitem{Fuel pressure}{check}
    \checkitem{Oil pressure \& oil temp.}{check}
    \checkitem{Alternate static}{check}
    \checkitem{Annunciator panel}{check}
    \checkitem{Throttle}{idle}
    \checkitem{Mixture}{lean for taxi}
    % TODO: The Duchess POH mentions a procedures for testing the alternators.
\end{checklist}

\begin{checklist}{Before Takeoff}
    \checkitem{Flight controls}{check}
    \checkitem{Flight instruments}{check}
    \checkitem{Carburetor heat}{off}
    \checkitem{Cowl flaps}{set}
    \checkitem{Flaps}{set}
    \checkitem{Trim}{set}

    \begin{center}
        \emph{\hypertarget{departure-briefing}{Departure briefing}}
    \end{center}

    \checkitem{\hyperlink{runway-length-table}{Takeoff distance}}{briefed}
    \checkitem{Terrain \& obstacles}{briefed}
    \checkitem{Takeoff minimums}{briefed}
    \checkitem{Departure procedure}{briefed}

    \begin{center}
        \emph{Abnormal operations}
    \end{center}

    \checkitem{Engine power loss}{briefed}
    \centering{
        (before \& after gear up)
        \\
    }
\end{checklist}

\begin{checklist}{Takeoff}
    \checkitem{Time off}{noted}
    \checkitem{Doors \& windows}{secured}
    \checkitem{Exterior lights}{set}
    \checkitem{Fuel pump}{on}
    \checkitem{Mixture}{full rich or target EGT}
    \checkitem{Prop}{full forward}
    \checkitem{Throttle}{full power}

    \begin{tcolorbox}[boxsep=0mm,left=0mm,right=0mm,colframe=black,colback=black,sharpish corners] 
        \color{white}
        \centering {
            \textbf{AIRSPEED: BLUELINE.\\}
        }
    \end{tcolorbox}
\end{checklist}

\begin{checklist}{Cruise}
    \checkitem{Power}{set}
    \checkitem{Fuel pump}{off}
    \checkitem{Mixture}{target EGT}
\end{checklist}

\begin{checklist}{Before Approach}
    \checkitem{NOTAMS}{briefed}
    \checkitem{ATIS, arrival, \& approach}{briefed}
    \checkitem{Terrain \& taxi}{briefed}
    \checkitem{Specials}{briefed}
\end{checklist}

\begin{checklist}{Approach}
    \checkitem{Altimeter}{verify}
    \checkitem{DA or MDA}{verify MSL}
    \checkitem{Landing gear}{down}
    \checkitem{Throttle}{17$''$}
    \checkitem{Prop}{2300 RPM}
    \checkitem{Airspeed}{100 KIAS}
    \checkitem{Mixture}{constant EGT}
\end{checklist}

\begin{checklist}{After Landing}
    \checkitem{Flaps}{retract}
    \checkitem{Mixture}{lean for taxi}
    \checkitem{Fuel pump}{off}
    \checkitem{Carburetor heat}{off}
    \checkitem{Cowl flaps}{set}
\end{checklist}

\begin{checklist}{V-Speeds (3,900 lbf)}
    \checkitem{$V_{BG}$}{95 KIAS}
    \checkitem{$V_{R}$}{71 KIAS}
    \checkitem{$V_{X}$}{71 KIAS}
    \checkitem{$V_{XSE}$}{85 KIAS}
    \checkitem{$V_{Y}$}{85 KIAS}
    \checkitem{$V_{YSE}$}{85 KIAS}
    \checkitem{$V_{CC}$}{100 KIAS}
    \checkitem{$V_{LE}$}{140 KIAS}
    \checkitem{$V_{LO(RET)}$}{112 KIAS}
    \checkitem{$V_{Ref}$}{76 KIAS}
    \checkitem{$V_{A}$}{132 KIAS}
    \checkitem{$V_{MCA}$}{65 KIAS}
    \checkitem{$V_{S_{0}}$/$V_{S_{1}}$}{58/68 KIAS}
\end{checklist}

\begin{checklist}{Engine Shutdown}
    \checkitem{Avionics bus 1 \& 2}{off}
    \checkitem{Lights}{off}
    \checkitem{Throttle}{1000 RPM}
    \checkitem{Mixture}{idle cut-off}
    \checkitem{Ignition}{off}
    \checkitem{Bat. and alt. switches}{off}
\end{checklist}

\onecolumn

\phantomsection
\addcontentsline{toc}{section}{Tables and Figures}

\phantomsection
\addcontentsline{toc}{subsection}{Takeoff Distance Over a 50' Obstacle At 3,600 lbf}

\begin{table}[H]
    \caption{Takeoff Distance Over a 50' Obstacle At 3,600 lbf}

    \begin{center}
        \begin{tabular}{ccc}
            \toprule
            \textbf{P\textsubscript{alt}} & \textbf{OAT (\textdegree{C})} & \textbf{Distance (ft)} \\
            \midrule
            1,000' & 15 \textdegree{C}               & 2,000'          \\
            5,000' &  40 \textdegree{C}                 & 5,000'              \\
            &                  &              
        \end{tabular}
    \end{center}
\end{table}

\phantomsection
\addcontentsline{toc}{subsection}{Landing Distance Over a 50' Obstacle At 3,600 lbf}

\begin{table}[H]
    \caption{Landing Distance Over a 50' Obstacle At 3,600 lbf}
    \begin{center}
        \begin{tabular}{ccc}
            \toprule
            \textbf{P\textsubscript{alt}} & \textbf{OAT (\textdegree{C})} & \textbf{Distance (ft)} \\
            \midrule
            1,000' & 15 \textdegree{C}               & 1,970'          \\
            &                  &              
        \end{tabular}
    \end{center}
\end{table}


\subsection{Climb Notes}

\begin{itemize}
    \item{At 3,900 lbf around 5,000' D\textsubscript{alt}, you should see 1,000 fpm with AEO.}
    \item{At 3,600 lbf around 1,000' D\textsubscript{alt}, you should see $\approx$ 250 fpm with OEI.}
    \item{You're not even hitting 200'/NM. Maybe at $V_Y$ between 0-1000'. Not enough to make it to the enroute environment IFR.}
\end{itemize}

\pagebreak

\phantomsection
\addcontentsline{toc}{subsection}{Rate of climb/descent (ft. per min)}

% https://aviation.stackexchange.com/questions/26322/how-does-the-faa-compute-the-values-in-tpps-climb-descent-table
\newcommand{\fpmatangle}[2]{\fpeval{round((#1/60) * tand(#2) * 6076.1)}}
\newcommand{\fpmatanglerounded}[2]{\fpeval{round(\fpmatangle{#1}{#2} / 5) * 5}}

\begin{table}[H]
    \caption{Rate of climb/descent (ft. per min)}

    \begin{center}
        \begin{tabular}{ccccccc}
            \toprule
            \textbf{ft/NM} & \multicolumn{5}{c}{\textbf{Ground speed (knots)}} & \textbf{Angle}
            \\\cmidrule(lr){2-6}
                           & 60                                                & 75                          & 90                          & 105                          & 120                          &                  \\
            \midrule
            210            & \fpmatanglerounded{60}{2.0}                       & \fpmatanglerounded{75}{2.0} & \fpmatanglerounded{90}{2.0} & \fpmatanglerounded{105}{2.0} & \fpmatanglerounded{120}{2.0} & 2.0\textdegree{} \\
            318            & \fpmatangle{60}{3.0}                              & \fpmatangle{75}{3.0}        & \fpmatangle{90}{3.0}        & \fpmatangle{105}{3.0}        & \fpmatangle{120}{3.0}        & 3.0\textdegree{} \\
            530            & \fpmatanglerounded{60}{5.0}                       & \fpmatanglerounded{75}{5.0} & \fpmatanglerounded{90}{5.0} & \fpmatanglerounded{105}{5.0} & \fpmatanglerounded{120}{5.0} & 5.0\textdegree{} \\
            745            & \fpmatanglerounded{60}{7.0}                       & \fpmatanglerounded{75}{7.0} & \fpmatanglerounded{90}{7.0} & \fpmatanglerounded{105}{7.0} & \fpmatanglerounded{120}{7.0} & 7.0\textdegree{} \\
            \bottomrule
        \end{tabular}
    \end{center}
\end{table}

\phantomsection
\addcontentsline{toc}{subsection}{Additional runway length required to clear low, close-in obstacles}

\begin{table}[H]
    \hypertarget{runway-length-table}{\caption{Additional runway length required to clear low, close-in obstacle}}

    \begin{center}
        \begin{tabular}{lccc}
            \toprule
                                   & \multicolumn{3}{c}{\textbf{Climb Angle}}
            \\\cmidrule(lr){2-4}
                                   & 745'/NM                                  & 530'/NM                                 & 318'/NM                                 \\
            \midrule
            % We subtract 50' from the obstacle height because we assume
            % takeoff performance data clears an initial 50' obstacle.
            \textbf{200' obstacle} & \num{\fpeval{ceil((6076*150)/745, 0)}}'  & \num{\fpeval{ceil((6076*150)/530, 0)}}' & \num{\fpeval{ceil((6076*150)/318, 0)}}' \\
            \textbf{150' obstacle} & \num{\fpeval{ceil((6076*100)/745, 0)}}'  & \num{\fpeval{ceil((6076*100)/530, 0)}}' & \num{\fpeval{ceil((6076*100)/318, 0)}}' \\
            \textbf{100' obstacle} & \num{\fpeval{ceil((6076*50)/745, 0)}}'   & \num{\fpeval{ceil((6076*50)/530, 0)}}'  & \num{\fpeval{ceil((6076*50)/318, 0)}}'  \\
            \bottomrule
        \end{tabular}
    \end{center}

    \textbf{Note:}
    \begin{itemize}
        \item Assumes takeoff performance data is based on clearing a 50' obstacle.
        \item Subtract obstacle's distance from runway end from required runway length.
        \item \hyperlink{departure-briefing}{Return back to the departure briefing.}
    \end{itemize}
\end{table}



\end{document}
